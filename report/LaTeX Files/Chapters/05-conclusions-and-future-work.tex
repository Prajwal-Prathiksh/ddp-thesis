% Chapter 5

\chapter{Conclusion \& Future Work} % Main chapter title
\label{chap:conclusions-and-future-work}

Werner Heisenberg, an eminent theoretical physicist of the $20^{th}$ century and one of the pioneers of quantum mechanics, said -
\begin{displayquote}
    "When I meet God, I am going to ask him two questions: why relativity? And why turbulence? I really believe he will have an answer for the first."
\end{displayquote}
His tongue-in-cheek quip regarding turbulence seems relevant even today, given that the comprehensive theory of turbulence still eludes us. This is a testament to the complexity of dealing with turbulence from a theoretical and numerical perspective. 

Marking the end of this project, this chapter summarises the work done and the conclusions drawn from the results obtained. It also discusses the future work that can be done to further the scope of this project.
A key component of this project, invloved compiling a comprehensive database of past SPH turbulence models. Each of the models have been listed, with their governing equatios, SPH discretisations and its derivation, if available. The limitations of each model have also been discussed, as well as potential areas for refinement, and any advantages that they provide when dealing with s specific class of problems.

Efforts to extend the EDAC scheme \parencite{Ramachandran2019} with the Lagrangian LES model is rendered challenging as detailed in \secref{sec:Extension-to-EDAC-SPH-Scheme}. The scheme also cannot make use of RANS-based models since the averaging technique assumes incompressibility, which simplifies to $(\nabla \cdot \vect{v} = 0)$, which would affect the transport equation for pressure \Eqref{eq:Ramachandran2019-edac-p-transport} which contains a term dependent of the divergence of velocity.
Therefore, a potential avenue of future work would be to consider the compressible form of EDAC \parencite{Chola2021} and proceed to develop a suitable turbulence for the scheme. This would be a fruitful endeavour since the scheme typically produces a smoother and more accurate pressure distribution for flows, confined or free, without requiring artificial viscosity. This would be a significant improvement over the current state-of-the-art SPH schemes, which are typically plagued by the presence of spurious pressure oscillations.

The project, further led to a better understanding of post-processing techniques available for SPH problems to study the effect of turbulence and evolution of the flow fields.
Characterisation of the energy spectrum of the flow, which is a standard technique employed in the FEM/FVM community, had been a challenging task for the particle-based Lagrangian method of SPH. 
This project, through a comparative study of various interpolation techniques, kernel type, kernel radius scale, particle resolution and the amount of disorder in their spacing, allowed for appropriate recommendations to be made. This resulted in the development of a \texttt{TurbulentFlowApp} class for the \texttt{PySPH} framework, which can be used to study the energy spectrum of the flow.
Similar work was attempted to characterise the FTLE fields between two time-instances of the flow. However, the results obtained revealed that this at best provide a qualitative measure of the attracting/repelling pathlines which could only be used for visualisation purposes, and possibly not for quantitative analysis. The method of using tracer particles, is suggested as a potential alternative which can be explored in future. They have the advantage of being able to be advected by the flow, and can store the exact forces acting on them, which can be used to quantitatively analyse the flow.

With the prcessing aspect of the project complete, the next stage of the project invloved identifying representative schemes from each class of the five major turbulence models, and implementing them in \texttt{PySPH}.
This allowed a detailed study of the performance of each of the models. Potential refinements in terms of scheme-specific parameters were also identified, along with improvements that could be made to the SPH discretisation. These results have been compiled in \tabref{tab:opt-sph-schemes}.
The comparative study invloved studying the schemes at three levels of Reynolds number regimes $Re: [10^2, 10^3, 10^4]$, at multiple resolution scales. This made a study on the convergence order of these schemes possible. Furthermore, through the means of the energy spectrum, and flow field visualisation, a detailed analysis of the schemes' performance has been compiled. It was also instrumental in identifying the issues with the schemes' implementation as decribed solely from literature.
This goes to show that the aspect of reproducing the results of a scheme from literature is not a trivial task, and requires a detailed understanding of the scheme, and the ability to identify the key parameters that affect the performance of the scheme. This task, is not made any easier by the fact that generally the codes are not made available, along with the fact that the schemes are not implemented in a standardised manner.
This also happens to be an area of concern with SPH, which unlike its typical CFD counterparts, lacks standardisation and is difficult to reproduce the results from research literature. This could prove to be a hurdle, in the widespread adoption of SPH as a viable alternative to FEM/FVM especially in the field of turbulence modelling.

Once the schemes, were optimised, the project then invloved a long-time simulation analysis to evaluate the performance of these schemes, based on identifying potential instabilities that could creep into the flow from building of numerical errors, or issues with the physical modelling. This study involved studying the TGV problem at two levels of Reynolds number regimes $Re: [10^4, 10^5]$, across two levels of finer resolution $N: [100^2, 200^2]$.
The work 


Subsequently, once a class of well-equipped turbulent models have been identified, it is intended to use those models in conjunction with a second-order convergent scheme of SPH, such as the L-IPST-C scheme devised by Negi and Ramachandran \parencite{Negi2022Techniques}. 

Following that, the subsequent stage of the project can delve into incorporating adaptive particle refinement to reduce the computational cost for the flow simulation for a given quality of the solution. Work done by Muta and Ramachandran \parencite{Muta2022} will be considered to make this concept of "adaptive turbulence" modelling feasible.

Achieving these objectives would allow work to be done on incorporating boundary conditions and wall functions for turbulence modelling as detailed by Mayrhofer \parencite{Mayrhofer2014}. 
This would therefore allow for simulations involving moving or deformable geometries, in which case the issue of the computational cost would take precedence. The work of Haftu et al. \parencite{Haftu2022} on parallel adaptive WCSPH and Negi et al. \parencite{negi2020improved} on inlet-outlet boundary conditions would serve as a helpful junction to proceed in future.
