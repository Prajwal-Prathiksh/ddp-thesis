\chapter{Introduction} % Main chapter title
\label{Chapter1}

Turbulent flows, has long been a phenomenon which is surprisingly easy to detect and observe in the natural world, but unmistakably challenging to understand and model in sufficient detail unlike other problems in classical physics.
The fundamental aspects of turbulent flow consisting of eddies of various length scales had long been observed, as recorded by Leonardo Da Vinci's $16^{th}$ century diagrams of water flow in streams and channels from the  \parencite{Colagrossi2021DaVinci}.
It took the work of Osborne Reynolds and William Thomson (Lord Kelvin) from the late $19th$ century, for the scientific community to clearly demarcate the separation between laminar and turbulent flow. However, further analysis of the subject remained arduous despite the NS equations having been written down in the early $19^{th}$ century. There is consensus that this remarkable failure of some of the greatest scientific minds in providing an intricate understanding of turbulence, points to an inadequacy of the mathematical tools we have developed in dealing with the strong non-linearity of the equations coupled with the tendency of flows to degenerate into some form instability.


Turbulence is caused by excessive kinetic energy in parts of a fluid flow, which overcomes the damping effect of the fluid's viscosity

The onset of turbulence can be predicted by the dimensionless Reynolds number, the ratio of kinetic energy to viscous damping in a fluid flow. However, turbulence has long resisted detailed physical analysis, and the interactions within turbulence create a very complex phenomenon.


Flows qualified as “turbulent” can be found in most fields that make use of
fluid mechanics. These flows posses a very complex dynamics whose intimate
mechanisms and repercussions on some of their characteristics of interest to
the engineer should be understood in order to be able to control them. The
criteria for defining a turbulent flow are varied and nebulous because there
is no true definition of turbulence. Among the criteria most often retained,
we may mention [150]: % Cousteix, J. (1989): Turbulence et couche limite (in french). CEPADUES – Editions, France
– the random character of the spatial and time fluctuations of the velocities, which reflect the existence of finite characteristic scales of statistical
correlation (in space and time);
– the velocity field is three-dimensional and rotational;
– the various modes are strongly coupled, which is reflected in the nonlinearity of the mathematical model retained (Navier–Stokes equations);
– the large mixing capacity due to the agitation induced by the various scales;
– the chaotic character of the solution, which exhibits a very strong dependency on the initial condition and boundary conditions.


\section{Project Motivation}

\section{Research Aims \& Objectives}

\section{Report Structure}
