% Chapter 5

\chapter{Conclusion \& Future Work} % Main chapter title
\label{chap:conclusions-and-future-work}

Werner Heisenberg, an eminent theoretical physicist of the $20^{th}$ century and one of the pioneers of quantum mechanics, said -
\begin{displayquote}
    "When I meet God, I am going to ask him two questions: why relativity? And why turbulence? I really believe he will have an answer for the first."
\end{displayquote}
His tongue-in-cheek quip regarding turbulence seems relevant even today, given that the comprehensive theory of turbulence still eludes us. This is a testament to the complexity of dealing with turbulence from a theoretical and numerical perspective. 

At the interim stage of this project, a reasonable database of past SPH turbulence models has been compiled, with their corresponding advantages, limitations and potential areas for refinement.
Efforts to extend the EDAC scheme \parencite{Ramachandran2019} with the Lagrangian LES model is rendered challenging as detailed in \secref{sec:Extension-to-EDAC-SPH-Scheme}. The scheme also cannot make use of RANS-based models since the averaging technique assumes incompressibility, which simplifies to $(\nabla \cdot \vect{v} = 0)$, which would affect the transport equation for pressure \Eqref{eq:Ramachandran2019-edac-p-transport} which contains a term dependent of the divergence of velocity. It is, therefore, worthwhile to consider the compressible form of EDAC \parencite{Chola2021} and proceed to develop a suitable turbulence for the scheme. This would be a fruitful endeavour since the scheme typically produces a smoother and more accurate pressure distribution for flows, confined or free, without requiring artificial viscosity.

Subsequently, once a class of well-equipped turbulent models have been identified, it is intended to use those models in conjunction with a second-order convergent scheme of SPH, such as the L-IPST-C scheme devised by Negi and Ramachandran \parencite{Negi2022Techniques}. 

Following that, the subsequent stage of the project can delve into incorporating adaptive particle refinement to reduce the computational cost for the flow simulation for a given quality of the solution. Work done by Muta and Ramachandran \parencite{Muta2022} will be considered to make this concept of "adaptive turbulence" modelling feasible.

Achieving these objectives would allow work to be done on incorporating boundary conditions and wall functions for turbulence modelling as detailed by Mayrhofer \parencite{Mayrhofer2014}. 
This would therefore allow for simulations involving moving or deformable geometries, in which case the issue of the computational cost would take precedence. The work of Haftu et al. \parencite{Haftu2022} on parallel adaptive WCSPH and Negi et al. \parencite{negi2020improved} on inlet-outlet boundary conditions would serve as a helpful junction to proceed in future.
