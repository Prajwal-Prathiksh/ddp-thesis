\chapter{Flow past a uniform cylinder} % Main chapter title
\label{Chapter2}

\section{Inviscid Flow}

When we consider an inviscid flow past a circular cylinder (without any circulation), under the assumptions of Potential Flow Theory, the stream function ($\psi$) is defined as follows in the polar coordinate system \parencite{White2018-ai}:

\begin{equation}
	\psi = \bigg(r - \frac{R^{2}}{r}\bigg) V_{\infty} \sin(\theta)
\end{equation}
Where ($R$) is the radius of the cylinder and ($V_{\infty}$) is the free-stream velocity. Therefore, the velocity components are defined as:
\begin{equation}
	V_r = \frac{1}{r} \frac{d \psi}{d \theta}=  \bigg(1 - \frac{R^{2}}{r^2}\bigg) V_{\infty} \cos(\theta)
\end{equation}
\begin{equation}
	V_{\theta} = - \frac{d \psi}{d r} =  - \bigg(1 + \frac{R^{2}}{r^2}\bigg) V_{\infty} \sin(\theta)
\end{equation}
Whereas for an inviscid flow past a circular cylinder with circulation, the stream function is as follows:
\begin{equation}
	\psi = \bigg(r - \frac{R^{2}}{r}\bigg) V_{\infty} \sin(\theta) - K\ln\big(\frac{r}{R}\big)
\end{equation}
Where ($K$) is the strength of the vortex at the center of the cylinder.


\section{Viscous Flow}

\subsection{Dimensional Analysis}
Let us being by assuming that the Drag ($D$) is a function of the free-stream velocity ($V_{\infty}$), the radius of cylinder ($R$), viscosity ($\mu$), vortex shedding frequency ($f$), shear stress on the surface of the cylinder ($\tau$), the density of the fluid ($\rho$) and the boundary layer thickness ($\delta$).
\begin{equation}
	D = \mathbbm{f} \big( V_{\infty}, R, \mu, f, \tau, \rho, \delta \big)
\end{equation}
By reducing this equation using the step-by-step method by Ipsen \parencite{ipsen1960units}, we arrive at the following dimensionless $\Pi$ groups.
\begin{equation}
	\Bigg( \frac{D}{\rho V_{\infty}^2 R^2} \Bigg) = \mathbbm{f} \Bigg( \frac{\mu}{\rho V_{\infty} R}, \frac{R f}{V_{\infty}}, \frac{\tau}{\rho V_{\infty}^2}, \frac{\delta}{R} \Bigg)
\end{equation}
Here the $\Pi$ groups are as follows:
\begin{itemize}
	\item $\Pi_1 = \frac{D}{\rho V_{\infty}^2 R^2} \rightarrow$ Drag Coefficient
	\item $\Pi_2 = \frac{\rho V_{\infty} R}{\mu} \rightarrow$ Reynolds Number
	\item $\Pi_3 = \frac{2 R f}{V_{\infty}} \rightarrow$ Strouhal Number
	\item $\Pi_4 = \frac{\tau}{\rho V_{\infty}^2} \rightarrow$ Skin Friction Coefficient
	\item $\Pi_5 = \frac{\delta}{R}$
\end{itemize}
Here we see that the drag coefficient (or the thrust coefficient thereof) is related to the standard dimensionless constants of fluid mechanics.

\subsection{Vorticity generation from Boundary-Layer Theory}
% TODO - Remove unnecessary parts
By solving the Blasius equation a flow past a flat-plate \parencite{White2018-ai}, we know the relationship between $\delta$ and the Reynolds Number as:
\begin{equation}
	\frac{\delta}{x} \approx \frac{5.0}{Re_x^{1/2}}
\end{equation}
Now by considering the definition of vorticity, in the case of flow past a flat-plate we can reasonably make the following assumption:
\begin{equation}
	\vec{\omega} = \nabla \times \vec{V} = \bigg( \frac{\partial V_y}{\partial x} - \frac{\partial V_x}{\partial y} \bigg) \hat{z}
\end{equation}
\begin{equation}
	\Bigg(\frac{\partial V_x}{\partial y} \approx \frac{\Delta V_x}{\Delta y} = \frac{V_{\infty}}{\delta} \Bigg) >> \Bigg(\frac{\partial V_y}{\partial x}  \approx \frac{\Delta V_y}{\Delta x} = 0 \Bigg)
\end{equation}
\begin{equation}
	\implies \vec{\omega} \approx - \frac{V_{\infty}}{\delta} \hat{z} = -\frac{V_{\infty} Re_x^{1/2}}{5.0 x} \hat{z} = - \frac{V_{\infty}^{3/2}}{5.0 x^{1/2}}\sqrt{\frac{\rho}{\mu}} \hat{z}
\end{equation}
\begin{equation}
	\therefore \vec{\omega} = - \frac{\kappa}{x^{1/2}}\hat{z} \text{ and } \kappa = \frac{V_{\infty}^{3/2}}{5.0}\sqrt{\frac{\rho}{\mu}}
	\label{eq:flat-plate-vorticity-approx}
\end{equation}
Clearly we can see from eq. \ref{eq:flat-plate-vorticity-approx} that vorticity is inversely proportional to $x^{1/2}$ and directly proportional  to $V_{\infty}^{3/2}$.

Now consider the vorticity transport equation, for the case of incompressible and isotropic fluids, with conservative body forces:
\begin{equation}
	\frac{D \vec{\omega}}{D t} = \frac{\partial \vec{\omega}}{\partial t} + (\vec{V}.\nabla)\vec{\omega} = (\vec{\omega}.\nabla)\vec{V} + \nu \nabla^2 \vec{\omega} 
	\label{eq:vorticity-transport-equation}
\end{equation}
\begin{equation}
	\nu = \frac{\mu}{\rho}
\end{equation}
Where ($\nu$) is the kinematic viscosity of the fluid.
When we consider eq. \ref{eq:vorticity-transport-equation} for 2D space, we see that since:
\begin{equation}
	\vec{V} = (V_x, V_y, 0) \text{ and } \frac{\partial}{\partial z} = 0
\end{equation}
\begin{equation}
	(\vec{\omega}.\nabla)\vec{V} = 0
\end{equation}
\begin{equation}
	\implies \frac{D \vec{\omega}}{D t} = \nu \nabla^2 \vec{\omega} = \nu\nabla^2 \omega_z \hat{z} = \nu \frac{\partial^2 \omega_z}{\partial x^2} \hat{z} \text{ , } \because \vec{\omega} = (0, 0, \omega_z)
	\label{eq:vorticity-transport-equation-2D-simplified}
\end{equation}
By substituting eqs. \ref{eq:flat-plate-vorticity-approx} and \ref{eq:vorticity-transport-equation-2D-simplified}, we arrive at the result:
\begin{equation}
	 \frac{D \vec{\omega}}{D t} = - \frac{3 \nu \kappa}{4 x^{5/2}} \hat{z}
	 \label{eq:omega-dot-flat-plate}
\end{equation}

\begin{figure}[H]
	\centering
	\includegraphics[width=14cm]{Figures/Vortex Shedding}
	\caption{Schematic of vortex shedding during flow past a cylinder}
	\label{fig:vortex-shedding}
\end{figure}

Subsequently, eq. \ref{eq:omega-dot-flat-plate} was assumed to model the region of $\theta = 90^{\circ}$ on a cylinder in inviscid flow, by treating the region to be similar to that of a flat-plate in a small neighborhood of $\theta = 90^{\circ}$. This assumption provided an equation which governed the rate of vorticity on top of the cylinder - the region from where vortices are typically shed. This equation, in addition to the following assumptions detailed below, an equation to model the vorticity of the shed vortex was derived:'
\begin{itemize}
	\item Generation of vorticity at the point of shedding is governed by the eq. \ref{eq:omega-dot-flat-plate}
	\item Strouhal number is approximately a constant ($\approx 0.22$) for wide ranges of Reynolds number ($10^2 < Re < 10^7$) \parencite{White2018-ai}:
	\begin{equation}
		St = \frac{2 R f}{V_{\infty}}
	\end{equation}
	\item Time taken for a vortex to grow and detach from the boundary = $t_{sep}$
	\begin{equation}
		t_{sep} = \frac{1}{f} = \frac{2 R}{V_{\infty} (St)}
	\end{equation}
\end{itemize}
The derived equation turned out to be a function of the angle of vortex separation. Hence to model this angle in terms of known variables, work by Wu et al. \parencite{Wu2004} was considered, wherein they studied the separation angles for 2D laminar flow upto a $Re$ of 290 through both 2D soap-film experiments and 2D numerical simulations, and subsequently proposed the following empirical relationship between $\theta_s - Re$ ($\theta_s$ is the angle measured clock-wise from the forward stagnation point on the cylinder, in degrees):
\begin{equation}
	\theta_s = 95.7 + 267.1 Re^{-1/2} - 625.9 Re^{-1} + 1046.6 Re^{-3/2}
\end{equation}
\begin{equation}
	\Delta \theta = \theta_s - 90^{\circ} = 5.7 + 267.1 Re^{-1/2} - 625.9 Re^{-1} + 1046.6 Re^{-3/2}
	\label{eq:delta-theta-re-relation}
\end{equation}
The empirical relation given in eq. \ref{eq:delta-theta-re-relation} is characterised to have a root-mean-square error of $4 \times 10^{-4}$ for $7 \leq Re \leq 200$. The empirical relationship between $\Delta \theta - Re$ has been plotted in fig. \ref{fig:delta-theta-vs-re}.
\begin{figure}[H]
	\centering
	\includegraphics[width=12cm]{Figures/delta-theta-vs-re}
	\caption{Empirical relationship between $\Delta \theta - Re$}
	\label{fig:delta-theta-vs-re}
\end{figure}

This allowed for a closed-form equation for the shed vorticity, through known flow quantities and the Reynolds number, which showed that the magnitude of circulation would be directly proportional to the Reynolds number.

However this entire exercise, built on the premise that the flow on the top-most point of a cylinder in an inviscid flow is similar to that of the flow past a flat plate, could not be substantiated through adequate mathematical rigour. Hence, the subsequent equations and their corresponding dependencies and results, could not be validated as well.
Hence, the work shifted towards calculating the forces that act inside a control volume containing a cylinder for various configurations of trailing vortices, and their evolution in time, which is outlined in the subsequent chapter.
