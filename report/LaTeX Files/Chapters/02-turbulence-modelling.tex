% Chapter 2

\chapter{Turbulence Modelling} % Main chapter title

\label{chap:turbulence-modelling}
%----------------------------------------------------------------------------------------
%	Section 1 - Viscosity-Based Models
%----------------------------------------------------------------------------------------
\section{Viscosity-Based Models}
Violeau et al. \parencite{VIOLEAU2002} were amongst the early pioneers who tried to incorporate a turbulence model in SPH. They came up with two techniques to tackle the problem of turbulence in a Lagrangian framework, which so far had been neglected till then in research, namely, the eddy viscosity model and a generalised Langevin model. For each of their techniques, they considered the following equation of state \Eqref{eq:violeau-eos}, continuity equation \Eqref{eq:violeau-continuity} and momentum equation \Eqref{eq:violeau-mom}, based on the work of \parencite{Monaghan1992}:   
\begin{equation}
    P_i = B \Bigg[ \bigg( \frac{\rho_i}{\rho_o} \bigg)^{\gamma} - 1 \Bigg] \quad , \quad B = \frac{\rho_o c_s^2}{\gamma}
    \label{eq:violeau-eos}
\end{equation}
Where, $(P, \rho)$ denotes pressure and density respectively. $(...)_i$ denotes the property of the $i^{th}$ SPH particle, $(...)_o$ denotes the reference quantities, and $(c_s)$ denotes the speed of sound.
\begin{equation}
    \LagDerivative{\rho_i} = \sum_j m_j \VIJ \cdot \DWIJ
    \label{eq:violeau-continuity}
\end{equation}
Where $[\LagDerivative{()}]$ denotes the Lagrangian derivative, and $(m, \vect{v})$ represent the mass and velocity respectively. Here we adopt the nomenclature of $(...)_{ij}$ to denote the difference of a specified property between the $i^{th}$ and $j^{th}$ SPH particle $[(...)_{i} - (...)_{j}]$, and $(\DWIJ)$ denotes $[\nabla_i W(\RIJ,h) = \frac{\RIJ}{\RtwoIJ[]} \PartialDerivative[r_i]{\WIJ}]$, where $(h)$ is the kernel smoothing length.
\begin{equation}
    \LagDerivative{\vect{v}_i} = - \sum_j m_j \bigg( \frac{P_i}{\rho_i^2} + \frac{P_j}{\rho_j^2} + \Pi_{ij} \bigg) \DWIJ + \vect{F}_i
    \label{eq:violeau-mom}
\end{equation}
Where $(\vect{F})$ is the external body force, and the viscous term $(\Pi)$ is defined as:
\begin{equation}
    \Pi_{ij} = - \frac{16\nu}{\rho_i + \rho_j} \frac{\VIJ \cdot \RIJ}{\RtwoIJ + \MachineEpsilon^2} 
    \label{eq:violeau-diffusion-term}
\end{equation}
Here $(\nu)$ is the kinematic viscosity, and $(\MachineEpsilon)$ is the machine epsilon.


\subsection{Eddy Viscosity Model}
\label{sec:eddy-visc-model}
The eddy viscosity model was devised as a first-order closure model, which consisted of a relationship between the Reynolds stress tensor and the mean velocity gradients. Therefore, the momentum equation is similar to the equation in \Eqref{eq:violeau-mom}, except that the kinematic viscosity is replaced by the eddy viscosity $(\nu_t)$, and the velocities are Reynolds-averaged. In the SPH formalism, the diffusion term occurring is therefore defined as given in \Eqref{eq:violeau-turbulent-diffusion-term}, with the eddy viscosity defined according to \Eqref{eq:violeau-eddy-viscosity}.
\begin{equation}
    \widetilde{\Pi}_{ij} = -8 \frac{\nu_{t, i} + \nu_{t, j}}{\rho_i + \rho_j} \frac{\RAProp{\vect{v}}_{ij} \cdot \RIJ }{\RtwoIJ + \MachineEpsilon^2}
    \label{eq:violeau-turbulent-diffusion-term}
\end{equation}
\begin{equation}
    \nu_t = L_m^2 \FrobeniusNorm{S} = L_m^2 \sqrt{\FrobeniusInnerProduct{S}{S}}
    \label{eq:violeau-eddy-viscosity}
\end{equation}
Where $\RAProp{\vect{v}}$ is Reynolds-averaged velocity, $L_m$ refers to the mixing length scales, and $(\tensor{S})$ is the strain-rate tensor $(=[1/2][\nabla \vect{v} + \nabla \vect{v}^T])$, where we adopt the nomenclature of $(\tensor{A})$ to represent a second-rank tensor field. Also, here $(\FrobeniusInnerProduct{A}{B})$ denotes the Frobenius inner product of $\tensor{A}$ and $\tensor{B}$, and $(\FrobeniusNorm{A})$ denotes the Frobenius norm of $\tensor{A}$ $( = \sqrt{\FrobeniusInnerProduct{A}{A}})$.
The SPH formulation for the mean velocity gradients are given in \Eqref{eq:violeau-mean-velocity-gradient}.
\begin{equation}
    \nabla \RAProp{\vect{v}}_i = - \frac{1}{\rho_i} \sum_j m_j \RAProp{\vect{v}}_{ij} \otimes \DWIJ
    \label{eq:violeau-mean-velocity-gradient}
\end{equation}
Where $(\otimes)$ denotes the tensor product.

On simulating Poiseuille flow for a high Reynolds number case, the authors could show that the velocity profile showed only a slight discrepancy with theory, with the expected log-law profile near the walls \figref{fig:violeau2002-eddy-viscosity-result}. This indicated that the model is appropriate for turbulent mixing problems or for cases involving spatially-varying viscosity while restricted to shear flows.

\begin{figure}[H]
    \centering
    \includegraphics{Figures/research_papers/violeau2002-eddy-viscosity-result.png}
    \caption{Turbulent Poiseuille flow in a pipe $(Re = \SciNot{6.4}{4})$ modelled using the eddy viscosity model. Computed mean velocity profiles after $(t=1s)$ (solid circles), against theory (solid line). Reproduced from \cite{VIOLEAU2002}}
    \label{fig:violeau2002-eddy-viscosity-result}
\end{figure}

\subsection{Generalized Langevin Model}
Violeau et al. \parencite{VIOLEAU2002} also considered a stochastic approach, where the main idea is built on the concept of prescribing particle velocities as a random process, with properties fulfilling the theoretical turbulence hypotheses \parencite{pope1994lagrangi}. Hence, came about the Generalised Langevin model (GLM), where the particle acceleration is defined as:
\begin{equation}
    d\vect{v} = -\frac{1}{\rho} \nabla \RAProp{P} + \tensor{G}(\vect{v} - \RAProp{\vect{v}})dt + \sqrt{C_0 \epsilon dt}\Vec{\xi}
    \label{eq:violeau-glm-particle-accel}
\end{equation}
Where $\Vec{\xi}$ is a random vector statistically non-correlated with velocities. The closure for this model was defined by specifying $\tensor{G}$ as:
\begin{equation}
    \tensor{G} = \HalfFrac C_1 \frac{\epsilon}{k}\tensor{I} + C_2 \nabla \RAProp{\vect{v}}
\end{equation}
Where $(k)$ is the turbulent kinetic energy, $(\epsilon)$ the dissipation rate, and $(C_i)$ being constants - $(C_1=1.8, C_2=0.6)$. $(\tensor{I})$, here, denotes the identity tensor.
By modelling turbulence as GLM in SPH, the momentum equation derived was given by:
\begin{equation}
    \LagDerivative{\vect{v}_i} = -\sum_j m_j \bigg( \frac{\RAProp{P}_i}{\rho_i^2} + \frac{\RAProp{P}_j}{\rho_j^2} \bigg) \DWIJ - \HalfFrac C_1 \frac{\epsilon_i}{k_i} \vect{v}'_i + C_2 \nabla \RAProp{\vect{v}}_i \cdot \vect{v}'_i + \sqrt{\frac{C_0 \epsilon_i}{\Delta t}} \Vec{\xi}_i
    \label{eq:violeau-mom-glm}
\end{equation}
\begin{equation}
    \RAProp{\vect{v}} = \sum_j \frac{m_j}{\rho_j}\vect{v}_j W_h (\vect{r}_j)
    \label{eq:violeau-ra-vel}
\end{equation}
Where the fluctuations are defined as $\vect{v}' = \vect{v} - \RAProp{\vect{v}}$, and the local values of turbulent kinetic energy and dissipation rate are:
\begin{align}
    \epsilon_i = 2 \nu_{t, i} + 
    \FrobeniusNorm{S_i}^2 \\
    k_i = \frac{\epsilon_i \nu_{t, i}}{C_{\mu}} \quad , \quad C_{\mu} = 0.009
    \label{eq:violeau-k-eps}
\end{align}

It is to be noted that the authors did not estimate the dissipation rate through the proper velocity gradients since the fluctuations of random velocities do not reproduce the small eddies.
The same test case as mentioned in \secref{sec:eddy-visc-model} was considered for the performance of GLM. 
The authors observed large fluctuations. They attributed the discrepancy to the mean operator being redefined as given by \Eqref{eq:violeau-ra-vel} instead of being a Reynolds average. In fact, by redefining the mean operator in such a fashion, they appeared to have constructed a rudimentary LES filter. 
As observed in \figref{fig:violeau2002-GLM-result}, the fluctuations have an order of magnitude of $k^{1/2}$. However, as claimed by the authors, unlike the eddy viscosity model, the GLM method can be used for different flows instead of being restricted to only shear flows.
\begin{figure}[H]
    \centering
    \includegraphics{Figures/research_papers/violeau2002-GLM-result.png}
    \caption{Turbulent Poiseuille flow in a pipe $(Re = \SciNot{6.4}{4})$ modelled using the generalised Langevin model. Computed mean velocity profiles after $(t=1s)$ (solid circles), against theory (solid line). Reproduced from \cite{VIOLEAU2002}}
    \label{fig:violeau2002-GLM-result}
\end{figure}


\subsection{mSPH}
Adami et al. \parencite{Adami2012} devised a model built on their observation of SPH simulations, wherein the absence of viscosity in typical SPH formulations produced purely noisy particle motion. At finite viscosities, the method would over-predict dissipation. Hence to counter this, they essentially ``modified'' (hence the name: Modified SPH [mSPH]) the momentum equation and the equation of state to advect the particles in order to homogenise the particle distribution, in turn stabilising the numerical scheme. They were also able to reduce the artificial dissipation in transitional flows.

The authors considered summation density (\Eqref{eq:Adami2012-summation-density}), which is a function of the volume of the respective SPH particle as given by \Eqref{eq:Adami2012-vol}, as opposed to evolving density through the continuity equation \parencite{hu2006multi}. The modified equation of state as given by \Eqref{eq:Adami2012-eos}, is equivalent to the classical SPH equation-of-state with $\gamma=1$.
\begin{equation}
    \Vol_i = \frac{1}{\sum_j \WIJ}
    \label{eq:Adami2012-vol}
\end{equation}
\begin{equation}
    \rho_i = \frac{m_i}{\Vol_i} = m_i\sum_j \WIJ
    \label{eq:Adami2012-summation-density}
\end{equation}
\begin{equation}
    P_i = c_s^2 (\rho_i - \rho_o)
    \label{eq:Adami2012-eos}
\end{equation}
Where $(\Vol)$ represents the volume of a particle.

The momentum equation, which provides the acceleration of the particle, is a function of just the gradient and viscous shear forces as given by \Eqref{eq:Adami2012-mom-governing}. The corresponding SPH formulation was derived as given by \Eqref{eq:Adami2012-mom-sph}, which built on the earlier work of Hu and Adams \parencite{hu2007incompressible}.
\begin{equation}
    \LagDerivative{\vect{v}} = -\frac{1}{\rho}\nabla P + \nu \Delta( \vect{v} ) + \vect{F}
    \label{eq:Adami2012-mom-governing}
\end{equation}
\begin{equation}
    \LagDerivative{\vect{v}_i} = -\frac{1}{m_i} \sum_j (\Vol^2_i + \Vol^2_j) \frac{P_i \rho_j + P_j \rho_i}{\rho_i + \rho_i} \DWIJ - \frac{\eta}{m_i} \sum_j (\Vol^2_i + \Vol^2_j) \frac{\VIJ}{\RtwoIJ[]}\DWIJ + \vect{F}_i
    \label{eq:Adami2012-mom-sph}
\end{equation}
Where $[\Delta ()]$ denotes the component-wise Laplacian operator, and $(\eta)$ is the dynamic viscosity.

This scheme takes advantage of the regularisation of the
particle motion stemming from the additional background pressure $(P_o = \rho_o c_s^2)$. The additional force exerted by the background pressure counteracts non-homogeneous particle distributions, therein reducing numerical dissipation.

The authors estimated the energy spectra of the flow simulations in order to analyse the results of their test cases, using first and second-order moving-least-squares (MLS) method \parencite{gossler2001moving} and its subsequent Fourier transform \parencite{frigo2005design}.
Their first test case, the $2D$ variant of the Taylor-Green Vortex (TGV) problem, involved $8\times 8$ counter-rotating vortices, requiring $64^2$ particles. They considered the viscosity to be zero.
As seen in the time evolution of the 
\begin{figure}[H]
    \centering
    \includegraphics[scale=0.7]{Figures/research_papers/adami2012-evolution-vel-field-tgv.png}
    \caption{Velocity vector plot at $t=2$ (left) and $t=30$ (right). $Re = \infty$. Reproduced from \cite{Adami2012} }
    \label{fig:adami2012-evolution-vel-field-tgv}
\end{figure}

The time evolution of the velocity field is given in \figref{fig:adami2012-evolution-vel-field-tgv}, where it can be observed that the $2D$ turbulence is characterised by merging and pairing of small vortices. The energy spectra given in \figref{fig:adami2012-energy-spectra-tgv} show that at low wave numbers, both interpolation schemes give the same results, but at high wave numbers, the results differ. The energy spectrum of the standard SPH has a linear slope of magnitude $m = 1$ in a log-log scale equivalent to a purely noisy velocity field. Theoretically, however, $2D$ turbulence has an energy cascade with a slope of $m = -3$ in the inertial range, which is reasonably predicted using mSPH.
\begin{figure}[H]
    \centering
    \includegraphics[scale=0.55]{Figures/research_papers/adami2012-energy-spectra-tgv.png}
    \caption{Comparison of energy spectra $t=10$. $+$ and $\times$ denote standard SPH results with quintic spline and MLS interpolation; $\circ$ and $\square$ denote mSPH results with quintic spline and MLS interpolation. Reproduced from \cite{Adami2012} }
    \label{fig:adami2012-energy-spectra-tgv}
\end{figure}

The second test case employed by the authors was that of the $3D$ TGV problem requiring $64^3$ particles for a wide range of Reynolds numbers. The dissipation rate of the flow simulations are shown in \figref{fig:adami2012-dissipation-re400} and \figref{fig:adami2012-dissipation-re3000}. It can be observed that the standard SPH is unable to simulate transitional flows due to excessive dissipation. In contrast, mSPH can reproduce the dissipation rate reasonably well. This implies that the corrected particle transport velocity is an analogous eddy-viscosity model on scales below the numerical resolution.
\begin{figure}[H]
    \centering
    \includegraphics[scale=0.7]{Figures/research_papers/adami2012-dissipation-re400.png}
    \caption{Dissipation rate at $Re = 400$ using DNS (solid line), Smagorinsky model (dashed line), standard SPH ($+$) and mSPH ($\circ$). Reproduced from \cite{Adami2012}}
    \label{fig:adami2012-dissipation-re400}
\end{figure}
\begin{figure}[H]
    \centering
    \includegraphics[scale=0.7]{Figures/research_papers/adami2012-dissipation-re3000.png}
    \caption{Dissipation rate at $Re = 3000$ using DNS (solid line), Smagorinsky model (dashed line) and mSPH ($\circ$). Reproduced from \cite{Adami2012}}
    \label{fig:adami2012-dissipation-re3000}
\end{figure}

%----------------------------------------------------------------------------------------
%	Section 2 - Large Eddy Simulation-based Models
%----------------------------------------------------------------------------------------
\section{Large Eddy Simulation-based Models}
\subsection{Implicit Pressure Poisson-based Models}
\label{sec:Implicit-Pressure-Poisson-based-Models}
Gotoh et al. \parencite{Gotoh2004} were amongst the first to integrate Large Eddy Simulation techniques with the SPH method. They derived this LES-SPH model, based on incompressible flow, to tackle the problem of reflection and transmission characteristics of regular waves by a partially immersed curtain-type breakwater. In order to compare the dissipation efficiencies, they considered the non-overtopping and overtopping cases of the problem.

The governing equations of the system were described as given by the continuity equation in \Eqref{eq:Gotoh2004-continuity-governing} and the momentum equation in \Eqref{eq:Adami2012-mom-governing}.
\begin{equation}
    \frac{1}{\rho} \LagDerivative{\rho} + \nabla \cdot \vect{v} = 0
    \label{eq:Gotoh2004-continuity-governing}
\end{equation}

The LES mass and momentum conservation equations for the flow were derived by filtering the respective equations using a spatial filter $\overline{(...)}$ to obtain their filtered counterparts as given by \Eqref{eq:Gotoh2004-continuity-filtered} and \Eqref{eq:Gotoh2004-mom-filtered} respectively.
\begin{equation}
    \frac{1}{\rho} \LagDerivative{\rho} + \nabla \cdot \vect{\overline{v}} = 0
    \label{eq:Gotoh2004-continuity-filtered}
\end{equation}
\begin{equation}
    \LagDerivative{\vect{\overline{v}}} = -\frac{1}{\rho}\nabla \overline{P} + \nu \Delta ( \vect{\overline{v}} ) + \frac{1}{\rho}\nabla\cdot\tensor{\tau} + \vect{F}
    \label{eq:Gotoh2004-mom-filtered}
\end{equation}
\begin{equation}
    \frac{1}{\rho} \tensor{\tau} = \vect{\overline{v}} \otimes \vect{\overline{v}} - \overline{\vect{v} \otimes \vect{v}}
    \label{eq:Gotoh2004-stress-tensor}
\end{equation}

The stress tensor $(\tensor{\tau})$ defined in \Eqref{eq:Gotoh2004-stress-tensor} is closed using Boussinesq’s Hypothesis as defined in \Eqref{eq:Gotoh2004-boussinesq}.
\begin{equation}
    \frac{1}{\rho} \tensor{\tau} = 2\nu_t \tensor{S} - \frac{2}{3}k\tensor{I}
    \label{eq:Gotoh2004-boussinesq}
\end{equation}

The turbulent eddy viscosity is estimated using a modified Smagorinsky model as given in \Eqref{eq:Gotoh2004-eddy-visc}. This allows wall effects to be incorporated into the model, which the authors required to tackle the problem they were working on.
\begin{equation}
    \nu_t = \min(C_s \Delta x, \kappa d_{wall})^2 \sqrt{2 \FrobeniusInnerProduct{S}{S}}
    \label{eq:Gotoh2004-eddy-visc}
\end{equation}
\begin{equation}
    C_s=0.1 \quad , \quad \kappa=0.4
\end{equation}
Where $(C_s)$ is the Smagorinsky constant, $(\kappa)$ is the von Karman constant, $d_{wall}$ is the normal distance of the particle to the closest wall, and $(\Delta x)$ is the inter-particle spacing.

The first term in \Eqref{eq:Gotoh2004-eddy-visc} dominates the flow far away from the solid wall, thereby recovering the standard Smagorinsky model. However, the second term dominates for flow close to the wall; hence, the eddy viscosity is a function of the particle distance to the wall. This overcomes the disadvantage of the standard Smagorinsky being over-dissipative inside the laminar layer.

In order to solve the system of equations and evolve them in time, the authors employed the Predictive-Corrective time integrator, similar to the two-step projection method of Chorin \parencite{chorin1968numerical}. The prediction stage is outlined by \Eqref{eq:Gotoh2004-predict-start} - \Eqref{eq:Gotoh2004-predict-end}, where $(\delta t)$ denotes the time step.
\begin{equation}
    \Delta \vect{v}_* = \bigg( \nu \Delta ( \vect{\overline{v}} ) + \frac{1}{\rho}\nabla\cdot\tensor{\tau} + \vect{F} \bigg) \Delta t
    \label{eq:Gotoh2004-predict-start}
\end{equation}
\begin{equation}
    \vect{v}_* = \vect{v}_t + \Delta \vect{v}_*
\end{equation}
\begin{equation}
    \vect{r}_* = \vect{r}_t + \vect{v}_* \Delta t
    \label{eq:Gotoh2004-predict-end}
\end{equation}

The correction stage is outlined by \Eqref{eq:Gotoh2004-correct-start} - \Eqref{eq:Gotoh2004-correct-end}. $(\overline{P})$ which is required to update the $(\vect{v}_{t+1})$ term is calculated implicitly from \Eqref{eq:Gotoh2004-correct-pressure-implicit}, which is based on the filtered continuity equation given by \Eqref{eq:Gotoh2004-continuity-filtered} and assuming incompressibility $\LagDerivative{\rho}=0$.
\begin{equation}
    \Delta \vect{v}_{**} = -\frac{1}{\rho} \nabla \overline{P}_{t+1} \Delta t
    \label{eq:Gotoh2004-correct-start}
\end{equation}
\begin{equation}
    \nabla \cdot \bigg( \frac{1}{\rho_*} \nabla \overline{P}_{t+1} \bigg) = \frac{\rho_o - \rho_*}{\rho_o \Delta t^2}
    \label{eq:Gotoh2004-correct-pressure-implicit}
\end{equation}
\begin{equation}
    \vect{v}_{t+1} = \vect{v}_* + \Delta \vect{v}_{**}
\end{equation}
\begin{equation}
    \vect{r}_{t+1} = \vect{r}_t + (\vect{v}_t + \vect{v}_{t+1})\frac{\Delta t}{2}
    \label{eq:Gotoh2004-correct-end}
\end{equation}

In order to solve the system of equations given by \Eqref{eq:Gotoh2004-predict-start} - \Eqref{eq:Gotoh2004-correct-end} in an SPH setting, the authors presented the following SPH formulation for the flow property. The fluid density is given using a simple summation density \Eqref{eq:Gotoh2004-summation-density}.
\begin{equation}
    \rho_i = \sum_j m_j\WIJ
    \label{eq:Gotoh2004-summation-density}
\end{equation}

The pressure gradient term is defined in \Eqref{eq:Gotoh2004-grad-p-sph} in a symmetric form.
\begin{equation}
    \bigg( \frac{1}{\rho} \nabla \overline{P} \bigg)_i = \sum_j m_j \bigg( \frac{\overline{P}_i}{\rho_i^2} + \frac{\overline{P}_j}{\rho_j^2} \bigg) \DWIJ
    \label{eq:Gotoh2004-grad-p-sph}
\end{equation}

The divergence of $\vect{v}$ is also defined symmetrically as given by \Eqref{eq:Gotoh2004-div-u-sph}.
\begin{equation}
    \nabla \cdot \vect{\overline{v}}_i = \rho_i \sum_j m_j \bigg(\frac{\vect{\overline{v}}_i}{\rho^2_i} + \frac{\vect{\overline{v}}_j}{\rho^2_j} \bigg) \cdot \DWIJ
    \label{eq:Gotoh2004-div-u-sph}
\end{equation}

The pressure Laplacian, defined in \Eqref{eq:Gotoh2004-p-laplacian-sph}, is formulated as a hybrid of a standard SPH first derivative with a finite difference approximation for the first derivative to aid particle pressure stability \parencite{cummins1999sph}.
\begin{equation}
    \nabla \cdot \bigg( \frac{1}{\rho} \nabla \overline{P} \bigg)_i = \sum_j m_j \frac{8}{(\rho_i + \rho_j)^2} \frac{\overline{P}_{ij} \RIJ \cdot \DWIJ }{\RtwoIJ}
    \label{eq:Gotoh2004-p-laplacian-sph}
\end{equation}

The divergence of the stress tensor is defined in \Eqref{eq:Gotoh2004-div-tau-sph}.
\begin{equation}
    \bigg( \frac{1}{\rho} \nabla \cdot \tensor{\tau} \bigg)_i = \sum_j m_j \Bigg( \frac{1}{\rho_i^2}\tensor{\tau_i} + \frac{1}{\rho_j^2}\tensor{\tau_j} \Bigg) \cdot \DWIJ
    \label{eq:Gotoh2004-div-tau-sph}
\end{equation}

Finally, the laminar stress term, consisting of the velocity Laplacian term, is defined as given by \Eqref{eq:Gotoh2004-vel-laplacian-sph}.
\begin{equation}
    \big(\nu \Delta(\vect{\overline{v}}) \big)_i = \sum_j m_j \frac{4(\eta_i + \eta_j)}{(\rho_i + \rho_j)^2}\frac{\vect{\overline{v}}_{ij} \RIJ \cdot \DWIJ}{\RtwoIJ}
    \label{eq:Gotoh2004-vel-laplacian-sph}
\end{equation}

The authors used this SPH-LES model to investigate the wave interaction with a partially immersed breakwater and compared the results with experimentally obtained values of a similar setup. Their computational domain was $2D$ populated by $\approx 1.2 \times  10^4$ particles.

\begin{figure}[H]
    \centering
    \includegraphics[scale=0.8]{Figures/research_papers/gotoh2004-wave-profile-result.png}
    \caption{Time sequences of computational and experimental wave profiles near curtain wall (overtopping). Reproduced from \cite{Gotoh2004}}
    \label{fig:gotoh2004-wave-profile-result}
\end{figure}

As observed in the comparative plots given in \figref{fig:gotoh2004-wave-profile-result}, the model proves to be accurate in tracking free surfaces of large deformation without numerical diffusion. The authors also observed the model’s capability to simulate turbulence and eddy vortices realistically near the curtain wall. However, the authors also conclude that a more refined turbulence model will be required for further accuracy in predicting flow involving wave interactions.

Building on the work mentioned above, Shao and Gotoh \parencite{Shao2005} performed a comparative study of SPH and the Moving Particle Semi-Implicit (MPS) method coupled with an LES model. They also validated these models against experimental data.

The filtered conservation equations which the authors considered were the same as given by \Eqref{eq:Gotoh2004-continuity-filtered} - \Eqref{eq:Gotoh2004-boussinesq}. However, they incorporated the standard Smagorinsky model \parencite{smagorinsky1963general} given by \Eqref{eq:Shao2005-eddy-visc} as opposed to the modified model \Eqref{eq:Gotoh2004-eddy-visc}.
\begin{equation}
    \nu_t = (C_s \Delta x)^2
    \label{eq:Shao2005-eddy-visc}
\end{equation}

The authors consider the same predictive-corrective scheme to evolve their system as detailed in \Eqref{eq:Gotoh2004-predict-start} - \Eqref{eq:Gotoh2004-correct-end}.
Similarly, they follow the same SPH formulation outlined in \Eqref{eq:Gotoh2004-summation-density} - \Eqref{eq:Gotoh2004-vel-laplacian-sph}. They do however, slightly modify the pressure and velocity Laplacian terms as given in \Eqref{eq:Shao2005-P-laplacian-sph} and \Eqref{eq:Shao2005-vel-laplacian-sph} respectively.
\begin{equation}
    (\nabla^2 P)_i = \sum_j m_j \frac{4}{\rho_i + \rho_j} \frac{P_{ij} \RIJ \cdot \DWIJ }{\RtwoIJ}
    \label{eq:Shao2005-P-laplacian-sph}
\end{equation}
\begin{equation}
    \big(\nu \Delta(\vect{v}) \big)_i = \sum_j m_j \frac{2(\nu_i + \nu_j)}{\rho_i + \rho_j}\frac{\VIJ \RIJ \cdot \DWIJ}{\RtwoIJ}
    \label{eq:Shao2005-vel-laplacian-sph}
\end{equation}

The authors validated this SPH-LES Model using experimental data from the experimental data corresponding to a solitary wave breaking on the beach \parencite{Synolakis1986}. Their computational domain was $2D$ and consisted of $\approx \SciNot{1.8}{4}$ particles. 
From the computed wave profiles shown in \figref{fig:shao2005-wave-profile-result}, it can be visually observed that there is reasonable agreement between the experimental and computation data. This verifies the model’s accuracy in tracking free surfaces with less or no numerical diffusion.
Furthermore, by performing a convergence study of the SPH-LES model using the dam-break problem, the authors could show that the scheme’s spatial and temporal accuracy is $O(\Delta t + \Delta x^{1.25})$.

\begin{figure}[H]
    \centering
    \includegraphics[scale=0.6]{Figures/research_papers/shao2005-wave-profile-result.png}
    \caption{Experimental and computational wave profiles by SPH and MPS model. Reproduced from \cite{Shao2005}}
    \label{fig:shao2005-wave-profile-result}
\end{figure}


\subsection{Explicit Pressure Equation of State-based Models}
\subsubsection{Standard Smagorinsky Model}

Rogers and Dalrymple \parencite{ROGERS2005}, similar to the work on SPH-LES modelling detailed in \secref{sec:Implicit-Pressure-Poisson-based-Models}, came up with an LES-type sub-particle-scale (SPS) formulation based on the weakly compressible assumption in order to develop a turbulence model for SPH.

The authors considered the mass and momentum conservation equations as already given in \Eqref{eq:Gotoh2004-continuity-governing} and  \Eqref{eq:Adami2012-mom-governing} respectively, along with the equation of state \Eqref{eq:violeau-eos}. However, for the value of $(B)$ in the state equation, the authors considered the definition given in \Eqref{eq:ROGERS2005-eos-b}:
\begin{equation}
    B' = 10 u_{max}
    \label{eq:ROGERS2005-eos-b}
\end{equation}
Where $(u_{max})$ is the maximum particle velocity.

They subsequently filtered the compressible conservation equations using Favre averaging as given by \Eqref{eq:ROGERS2005-favre-averaging}.
\begin{equation}
    \widetilde{f} = \frac{\overline{\rho f}}{\overline{\rho}}
    \label{eq:ROGERS2005-favre-averaging}
\end{equation}


The derived filtered conservation equations for mass and momentum are detailed in \Eqref{eq:ROGERS2005-continuity-filtered} and \Eqref{eq:ROGERS2005-mom-filtered}.
\begin{equation}
    \LagDerivative{\overline{\rho}} = -\overline{\rho} \nabla \cdot \vect{\widetilde{v}}
    \label{eq:ROGERS2005-continuity-filtered}
\end{equation}
\begin{equation}
    \LagDerivative{\vect{\widetilde{v}}} = - \frac{1}{\overline{\rho}}\nabla \overline{P} + \frac{1}{\overline{\rho}} (\nabla \cdot \overline{\rho \nu} \nabla) \vect{\widetilde{v}} + \frac{1}{\overline{\rho}}\nabla\cdot\tensor{\tau} + \vect{F}
    \label{eq:ROGERS2005-mom-filtered}
\end{equation}

Where the SPS stress tensor and turbulent eddy viscosity is given by \Eqref{eq:ROGERS2005-tau} and \Eqref{eq:ROGERS2005-turbulent-eddy-visc} respectively.
\begin{equation}
    \tensor{\tau} = \overline{\rho}\bigg(2\nu_t \tensor{S} - \frac{2}{3}\operatorname{tr}[\tensor{S}]\tensor{I}\bigg) - \frac{2}{3}\overline{\rho}C_I \overline{\Delta}^2 \tensor{I} \quad , \quad C_I = \SciNot{6.6}{-4}
    \label{eq:ROGERS2005-tau}
\end{equation}
\begin{equation}
    \nu_t = (C_s \Delta x)^2 \sqrt{2 \FrobeniusInnerProduct{S}{S}} \quad , \quad C_s = 0.12
    \label{eq:ROGERS2005-turbulent-eddy-visc}
\end{equation}

As for the SPH formulations of the aforementioned governing equations, the authors

The authors derived the SPH formulations of the aforementioned governing equations. The continuity equation takes the form as detailed in \Eqref{eq:violeau-continuity}. The pressure gradient term is given in \Eqref{eq:Gotoh2004-grad-p-sph}. The laminar stress term, consisting of the velocity Laplacian is given by \Eqref{eq:ROGERS2005-vel-laplacian-sph}, which itself was built on the work of Morris et al. \parencite{morris1997modeling} as given in \Eqref{eq:morris1997modeling-vel-laplacian-sph}. Finally the stress divergence is defined by \Eqref{eq:Gotoh2004-div-tau-sph}.
\begin{equation}
    \bigg( \frac{1}{\rho} (\nabla \cdot \eta \nabla) \vect{v} \bigg)_i = \sum_j m_j \frac{\nu (\rho_i + \rho_i)}{\rho_{ij}^2} \frac{\VIJ \RIJ \cdot \DWIJ}{\RtwoIJ + \MachineEpsilon^2}
    \label{eq:ROGERS2005-vel-laplacian-sph}
\end{equation}
\begin{equation}
    \bigg( \frac{1}{\rho} (\nabla \cdot \eta \nabla) \vect{v} \bigg)_i = \sum_j m_j \frac{(\eta_i + \eta_j) \VIJ}{\rho_i \rho_j} \bigg( \frac{1}{\RtwoIJ[]} \PartialDerivative[r_i]{\WIJ} \bigg) \quad , \quad \DWIJ = \frac{\RIJ}{\RtwoIJ[]} \PartialDerivative[r_i]{\WIJ}
    \label{eq:morris1997modeling-vel-laplacian-sph}
\end{equation}

The authors noted that the LES description of viscous effects in slightly compressible SPH could lead to unphysical behaviour at free surfaces due to density variations being magnified by the equation of state. The lack of artificial viscosity implies that such variations are not damped. They subsequently noted that averaging the density would ensure smooth and physically acceptable free surfaces, based on the work of Panizzo \parencite{panizzo2004physical}. Hence, they performed Shepard filtering of the density as defined in \Eqref{eq:ROGERS2005-rho-shepard-filter} every 40-time steps.
\begin{equation}
    \rho_i = \frac{\sum_j \rho_j \WIJ \Vol_j}{\sum_j \WIJ \Vol_j}
    \label{eq:ROGERS2005-rho-shepard-filter}
\end{equation}

The authors simulated the problem of a weakly plunging breaker in $2D$ and $3D$ to ascertain the performance and capability of the model. Their $2D$ computational domain consisted of $\approx \SciNot{1}{5}$ particles, with the $3D$ domain consisting $\approx \SciNot{2}{4}$ particles.
The authors could show that in the case of the $2D$ problem, the model could predict regions of high vorticity that persisted longer when compared to standard SPH utilising conventional artificial viscosity. The model also displayed the turbulent bore, which generated reverse breaking, leading to the downbursting-like phenomenon, as observed in experiments \parencite{kubo2001large}.
In the case of the $3D$ problem, the authors showed the model’s capability to capture near vertically-oriented eddies despite the lower resolution. 

Building on this work, Dalrymple and Rogers \parencite{Dalrymple2006} used this scheme on a wide variety of problems, ranging from $2D$ Green water overtopping, $2D$ waves on a beach, $3D$ dam break and $3D$ waves on a beach. The quantitative analysis of the results allowed the authors to conclude that the model is especially suited for problems involving splash or flow separation. The authors also warn about the model’s requirement of a large number of particles for sufficient resolution. That and the finite speed of sound stemming from the compressible flow implied that time steps had to $O(10^{-5}s)$. Hence, the authors remain cautiously optimistic about the model since the method performs well for smaller regions where the number of particles is reasonable. However, they believe that extended Boussinesq codes would more efficiently model larger domains.

\subsubsection{Modified Smagorinsky Model}
Canelas et al. \parencite{Canelas2016} constructed the wall-adapting local eddy viscosity (WALE) model to be incorporated in the SPH-LES scheme. They noted that studying turbulent flow fields required the identification of vortices themselves to study their interactions in the flow. They used the definition of Lagrangian Coherent Structures (LCS) to help capture these vortices. As a Lagrangian method, SPH is preferable for studying LCS since the technique provides the motion of individual fluid particles, thereby eliminating the need for expensive post-processing inherent to Eulerian solutions. 
However, they noted that typically employed SPS strategies for LES simulations, based on the standard Smagorinsky model, cannot correctly enforce wall conditions and non-vanishing stresses with laminar flows. Hence they devised the WALE model.

The authors consider the compressible Navier-Stokes (NS) along the continuity equation as their governing equation. They subsequently present the SPH formulation of the continuity equation as defined by \Eqref{eq:Canelas2016-continuity-sph}.
\begin{equation}
    \LagDerivative{\rho_i} = - \rho_i \sum_j m_j \VIJ \cdot \DWIJ
    \label{eq:Canelas2016-continuity-sph}
\end{equation}
The pressure gradient term is given in \Eqref{eq:Gotoh2004-grad-p-sph}. The laminar stress term, consisting of the velocity Laplacian, is given by \Eqref{eq:Canelas2016-vel-laplacian-sph}. 
\begin{equation}
    \big(\nu \Delta(\vect{v}) \big)_i = \sum_j m_j \frac{4 \nu}{\rho_i + \rho_j}\frac{\VIJ \RIJ \cdot \DWIJ}{\RtwoIJ}
    \label{eq:Canelas2016-vel-laplacian-sph}
\end{equation}

Finally the stress divergence is defined by \Eqref{eq:Gotoh2004-div-tau-sph}, with the stress tensor being defined by \Eqref{eq:Canelas2016-tau}
\begin{equation}
    \tensor{\tau} = \rho \bigg( 2\nu_t \tensor{S} - \frac{2 \nu_t}{3}\operatorname{tr}[\tensor{S}]\tensor{I} \bigg) - \bigg( \frac{4}{3} \rho C_I (\Delta x)^2 \FrobeniusInnerProduct{S}{S} \bigg) \tensor{I} \quad , \quad C_I = \SciNot{6.6}{-3}
    \label{eq:Canelas2016-tau}
\end{equation}

The WALE model redefines the turbulent eddy viscosity as given by \Eqref{eq:Canelas2016-turbulent-eddy-visc}.
\begin{equation}
    \nu_t = \rho (C_w \Delta x)^2 \frac{\FrobeniusInnerProduct{S^d}{S^d}^{3/2}}{\FrobeniusInnerProduct{S}{S}^{5/2} + \FrobeniusInnerProduct{S^d}{S^d}^{5/4}} \quad , \quad C_w=0.325
    \label{eq:Canelas2016-turbulent-eddy-visc}
\end{equation}
\begin{equation}
    \tensor{S^d} = \HalfFrac \bigg( (\nabla \vect{v})^2 + \big((\nabla \vect{v})^T\big)^2 \bigg) - \frac{1}{3} \operatorname{tr}[(\nabla \vect{v})^2] \tensor{I}
\end{equation}

In order to test their model, the authors considered an array of cylinders in fluid flow as shown in \figref{fig:Canelas2016-vel-field}, with a constant body force. They computed the vorticity field and the Finite-Time Lyapunov Exponents (FTLE) field to study the LCSs in the flow. These fields are shown in \figref{fig:Canelas2016-vorticity-field} and \figref{fig:Canelas2016-ftle-field} respectively.
 \begin{figure}[H]
    \centering
    \includegraphics{Figures/research_papers/Canelas2016-vel-field.png}
    \caption{Cylinder distribution and instantaneous velocity overlapped by velocity vectors. Velocity in $m/s$, at $t=20s$. The flow direction is from left to right. Reproduced from \cite{Canelas2016}}
    \label{fig:Canelas2016-vel-field}
\end{figure}
\begin{figure}[H]
    \centering
    \includegraphics{Figures/research_papers/Canelas2016-vorticity-field.png}
    \caption{Vorticity field. Vorticity in Hz, at $t = 20s$. Reproduced from \cite{Canelas2016}}
    \label{fig:Canelas2016-vorticity-field}
\end{figure}
\begin{figure}[H]
    \centering
    \includegraphics{Figures/research_papers/Canelas2016-ftle-field.png}
    \caption{FTLE field with negative integration time $(T = -0.4 s)$ (unstable manifolds - attracting LCS) at $t = 20s$. Reproduced from \cite{Canelas2016}}
    \label{fig:Canelas2016-ftle-field}
\end{figure}

Okraschevski et al. \parencite{Okraschevski2022}, building on the work of Hardy \parencite{hardy1982formulas}, show that SPH should be viewed as a Lagrangian quadrature technique for the governing equations of explicit LES, raising interesting implications for SPH as a method itself. Firstly, the kernel scale limits SPH’s physical resolution, rendering it unsuitable as a DNS alternative. Secondly, any deficits introduced below the kernel scale could be resolved by consideration of the stress term, from which structures above the kernel scale could benefit. The authors consider this second implication a working hypothesis to try and prove or disprove.

The authors use the compressible NS equations and subject them to a spatial average to obtain the filtered governing equations. They subsequently derive the SPH formulations of the mass and momentum conservation equations as given in \Eqref{eq:Adami2012-summation-density} and \Eqref{eq:Okraschevski2022-mom-sph}.
\begin{equation}
    \overline{\rho}_i\LagDerivative{\widetilde{\vect{v}}} = - \sum_j (\overline{P}_i + \overline{P}_j) \DWIJ \Vol_j + 2(2+n_{dim})\eta \sum_j \frac{\widetilde{\vect{v}}_{ij} \cdot \RIJ}{\RtwoIJ}\DWIJ \Vol_j + \nabla \cdot \tensor{\tau}
    \label{eq:Okraschevski2022-mom-sph}
\end{equation}
Where $(n_{dim})$ denotes the number of spatial dimensions for a given problem.

The equation of state is given by \Eqref{eq:Okraschevski2022-eos-sph}, with the SFS tensor being given by \Eqref{eq:Okraschevski2022-tau}.
\begin{equation}
    \overline{P}_i = P_o + K\bigg(\frac{\overline{\rho}_i}{\rho_o} - 1 \bigg)
    \label{eq:Okraschevski2022-eos-sph}
\end{equation}
Where, $(K)$ is a constant.
\begin{equation}
    \tensor{\tau} = 2\nu_t\overline{\rho}\tensor{S}
    \label{eq:Okraschevski2022-tau}
\end{equation}

The SPH formulation for the strain rate tensor is given by \Eqref{eq:Okraschevski2022-S-tensor-sph}, and the divergence of the SFS tensor being given by \Eqref{eq:Okraschevski2022-div-tau-sph}.
\begin{equation}
    \tensor{S}_i = - \sum_j \widetilde{\vect{v}}_{ij} \otimes \DWIJ \Vol_j
    \label{eq:Okraschevski2022-S-tensor-sph}
\end{equation}
\begin{equation}
    \bigg(\nabla \cdot \tensor{\tau}\bigg)_i =\sum_j \big( \tensor{\tau}_i + \tensor{\tau}_j \big) \DWIJ \Vol_j
    \label{eq:Okraschevski2022-div-tau-sph}
\end{equation}

The authors consider three models for the turbulent eddy viscosity:
\begin{itemize}
    \item \texttt{SMAG}: Standard Smagorinsky model
        \begin{equation}
            \nu_t = (C_s \Delta x)^2 \sqrt{2 \FrobeniusInnerProduct{S}{S}}
            \label{eq:Okraschevski2022-SMAG}
        \end{equation}
        
    \item \texttt{SIGMA}: $\sigma$-model \parencite{nicoud2011using}, where $(\sigma_i)$ denotes the $i^{th}$ singular value of the velocity gradient
        \begin{equation}
            \nu_t = (C_{\sigma} \Delta x)^2 \frac{\sigma_3 (\sigma_1 - \sigma_2) (\sigma_2 - \sigma_3)}{\sigma_1^2}
            \label{eq:Okraschevski2022-SIGMA}
        \end{equation}
    
    \item \texttt{SMAG-MCG}: Standard Smagorinsky model discretized in the Monaghan-Cleary-Gingold (MCG) form \parencite{cleary1999conduction}
        \begin{equation}
            \nabla \cdot \tensor{\tau} = 2(2+n_{dim}) \sum_j \overline{\rho}_i \overline{\rho}_j \frac{\nu_{t, i} + \nu_{t, j}}{\overline{\rho}_i + \overline{\rho}_j} \frac{\widetilde{\vect{v}}_{ij} \cdot \RIJ }{\RtwoIJ} \DWIJ \Vol_j
            \label{eq:Okraschevski2022-SMAG-MCG}
        \end{equation}
\end{itemize}

The authors show a correlation between the vorticity and SFS tensor. They subsequently conclude that the fluctuations in the vorticity, energy spectra and SFS tensor are correlated quantities that contain different information about the numerical dissipation dynamics.

The authors use the $3D$ TGV problem to study and investigate the proposed models. They consider various resolutions to study the problem, varying from $\approx 200^3 - 500^3$ particles.
\begin{figure}[H]
    \centering
    \includegraphics[scale=0.9]{Figures/research_papers/Okraschevski2022-fig-6.png}
    \caption{Comparison of quantitative metrics for different particle counts $N$. (a) \& (b) Temporal evolution of the density-weighted averaged kinetic energy. (c) \& (d) Temporal evolution of the averaged dissipation rate. Reproduced from \cite{Okraschevski2022}}
    \label{fig:Okraschevski2022-fig-6}
\end{figure}
\begin{figure}[H]
    \centering
    \includegraphics[scale=0.9]{Figures/research_papers/Okraschevski2022-fig-9.png}
    \caption{Backward FTLE at the plane $z =\pi$ for $N = 2563$ and $t = 14 s$ in the range $[11;14] s$. (a) Without explicit SFS model. (b) With the SMAG model. (c) With SIGMA model. (d) With the SMAG-MCG model. Reproduced from \cite{Okraschevski2022}}
    \label{fig:Okraschevski2022-fig-9}
\end{figure}
\begin{figure}[H]
    \centering
    \includegraphics[scale=0.9]{Figures/research_papers/Okraschevski2022-fig-11.png}
    \caption{Comparison of quantitative metrics for different $N = 256^3$ runs. (a) \& (b) Temporal evolution of the density-weighted averaged kinetic energy. (c) \& (d) Temporal evolution of the averaged dissipation rate. Reproduced from \cite{Okraschevski2022}}
    \label{fig:Okraschevski2022-fig-11}
\end{figure}

The \texttt{SMAG} model degrades the flow in terms of the averaged kinetic energy in \figref{fig:Okraschevski2022-fig-6} (b) and the corresponding dissipation rate in \figref{fig:Okraschevski2022-fig-6} (d). This negative attribute of the model outweighs the positive attribute of the model, in that cases with the SMAG model in \figref{fig:Okraschevski2022-fig-9} (b) appear less tattered and exhibit symmetry compared to the reference WCSPH solution in \figref{fig:Okraschevski2022-fig-9} (a).
From \figref{fig:Okraschevski2022-fig-11}, it becomes evident that with larger $(C_s)$, the averaged kinetic energy is increasingly reduced, which is also reflected by the dissipation rates. This indicates that the best choice corresponds to $(C_s \rightarrow 0)$.

On a concluding note, the authors claim that the subsonic turbulence captured by SPH can correctly represent the flow field up to the kernel scale but at a high cost in the best of cases. They also note that explicit SFS
models, at most, only lead to marginal improvement of LCS. However, regarding the inertial range dynamics, the dissipative SFS models remove kinetic energy, which is highly detrimental since a spectral energy deficit already characterises SPH. The authors attribute these drawbacks to the non-local character of the Lagrangian quadrature, which they explain through the concept of \textit{Particle Duality}. The concept states that the SPH particles must represent superfluid element approximants and fluid element surrogates simultaneously, leading to an unphysical increase in particle interaction distances. The authors, hence
prove Rennehen’s expectation that explicit SFS models in an SPH framework can only degrade the quality of the approximation for subsonic turbulent flow \parencite{rennehan2021mixing}, which raises concerns for turbulence modelling in SPH. However, the authors note there might be hope from SPH native models.
%TODO: Clean up the above part. Add more results?

%----------------------------------------------------------------------------------------
%	Section 3 - Lagrangian LES-based Models
%----------------------------------------------------------------------------------------
\section{Lagrangian LES-based Models}
Having reviewed turbulence models developed for SPH, Di Mascio et al. \parencite{DiMascio2017} concluded that LES models could serve as an excellent middle ground between RANS and DNS. This is because only the sub-grid scale eddies would be modelled, allowing for the larger eddies to be directly simulated from the NS equations. They also note that contemporary SPH-LES models use the Eulerian differential operators in an SPH formulation and therefore do not have a rigorous background. Therefore they propose a Lagrangian form of LES to tie in with the Lagrangian quadrature technique, that is, SPH. 

The authors consider a weakly compressible, barotropic fluid whose conservation equations for mass and density are given by \Eqref{eq:Gotoh2004-continuity-governing} and \Eqref{eq:DiMascio2017-mom-governing} respectively.
\begin{equation}
    \LagDerivative{\vect{v}} = -\frac{1}{\rho} + \nu \Delta(\vect{v}) + (\lambda' + \nu)\nabla(\nabla \cdot \vect{v})
    \label{eq:DiMascio2017-mom-governing}
\end{equation}
Where, $(\lambda' = \lambda/\rho)$ is the Lam\'e constant.
The equation of state is simply given as \Eqref{eq:DiMascio2017-eos}.
\begin{equation}
    P = F(\rho)
    \label{eq:DiMascio2017-eos}
\end{equation}
\begin{equation}
    \LagDerivative{\vect{r}} = \vect{v}
    \label{eq:DiMascio2017-r-governing}
\end{equation}

The authors subsequently define a Lagrangian filter $(\phi)$ with compact support of the form
\begin{equation}
    \phi = \phi\big(\TildeR_p(t)-\vect{y}, t-\tau  \big)
    \label{eq:DiMascio2017-filter}
\end{equation}

Where the filter is an even function in its arguments, and $(\TildeR_p(t))$ is the position of a material which moves with velocity 
\begin{equation}
    \TildeV(\TildeR, t) = \IntRThreeAndT \phi \big(\TildeR_p(t)-\vect{y}, t-\tau  \big) \vect{v}(\vect{y}, \tau)d\tau dV_y
\end{equation}

By using the filter as defined in \Eqref{eq:DiMascio2017-filter}, the filtered governing equations are as follows
\begin{equation}
    \LagDerivative{\TildeRho} = -\TildeRho \nabla \cdot \TildeV + \nabla \cdot (\TildeRho\TildeV - \widetilde{\rho \vect{v}})
\end{equation}
\begin{equation}
    \LagDerivative{\TildeV} = - \frac{\nabla \TildeP}{\TildeRho} + \nu \Delta(\TildeV) + (\lambda'+\nu)\nabla(\nabla\cdot\TildeV) - \nabla[\widetilde{(\rho)}-G(\TildeRho)] + \nabla\cdot\tensor{T}_l+\widetilde{\vect{v}\nabla\cdot\vect{v}}
\end{equation}
\begin{equation}
    \LagDerivative{\TildeR}=\TildeV
\end{equation}
\begin{equation}
    \TildeRho = F(\TildeP)
\end{equation}
\begin{equation}
    \tensor{T}_l = \TildeV\otimes\TildeV - \widetilde{\vect{v}\otimes\vect{v}}
\end{equation}
\begin{equation}
    G(\rho) = \int^{\rho} \frac{1}{s}\frac{dF}{ds}ds
\end{equation}

In order for the authors to be able to reinterpret the Lagrangian LES through SPH, they split the filter as follows
\begin{equation}
    \phi\big(\TildeR_p(t)-\vect{y}, t-\tau  \big) = W(\TildeR - \vect{y})\theta(t-\tau)
\end{equation}

This allows the author to provide a relationship between the various type of filtered quantities as given
\begin{equation}
    \EncAngBrk{f}(\TildeR, t) = \int_{\mathbb{R}^3} W ( \TildeR - \vect{y} ) f ( \vect{y}, t ) dV_y
\end{equation}
\begin{equation}
    \bar{f}(\vect{y}, t) = \int_{\mathbb{R}} \theta(t-\tau) f(\vect{y}, \tau) d\tau
\end{equation}
\begin{equation}
    \widetilde{f} = \EncAngBrk{\bar{f}}, \EncAngBrk{\bar{f}} \neq \bar{\EncAngBrk{f}}
\end{equation}

The authors, therefore, can derive the equations in an SPH formalism as given by
\begin{equation}
    \LagDerivative{\TildeRho} = -\TildeRho \EncAngBrk{\nabla\cdot\TildeV} + C_1 + C_2
\end{equation}
\begin{equation}
    C_1 = -\TildeRho\EncAngBrk{\nabla\cdot(\bar{\vect{v}}-\TildeV)}
\end{equation}
\begin{equation}
    C_2 = \nabla\cdot(\TildeRho\TildeV - \widetilde{\rho\vect{v}})
\end{equation}
\begin{equation}
    \LagDerivative{\TildeV} = -\frac{\EncAngBrk{\nabla\TildeP}}{\TildeRho} + \nu\EncAngBrk{\Delta(\TildeV)} + (\lambda'+\nu)\EncAngBrk{\nabla(\nabla\cdot\TildeV)} + M_1 + M_2
\end{equation}
\begin{equation}
    M_1 = -\frac{\EncAngBrk{\nabla(\bar{\rho}-\TildeRho)}}{\TildeRho} + \nu\EncAngBrk{\Delta(\bar{\vect{v}}-\TildeV)} + (\lambda'+\nu)\EncAngBrk{\nabla(\nabla\cdot(\bar{\vect{v}}-\TildeV))}
\end{equation}
\begin{equation}
    M_2 = -\nabla\big(\widetilde{G(\rho)}-G(\TildeRho)\big) + \widetilde{\vect{v}\nabla\cdot\vect{v}} + \nabla\cdot\tensor{T}_l
\end{equation}
\begin{equation}
    \LagDerivative{\TildeR} = \TildeV
\end{equation}
\begin{equation}
    \TildeP=F(\TildeRho)
\end{equation}

The authors close the terms $(C_2, M_2)$ as follows
% \begin{equation}
%     C_1 \approx \TildeRho\EncAngBrk[\bigg]{\nabla\cdot\big(\sigma^2\Delta(\TildeV)\big)}, \sigma^2=h^2\alpha
% \end{equation}

% Where $(\alpha)$ is a dimensionless parameter.
\begin{equation}
    C_2 \approx \nabla \cdot (\nu_{\delta} \nabla\TildeRho),
\end{equation}
\begin{equation}
    \nu_{\delta} = (C_{\delta}\sigma)^2\sqrt{2\FrobeniusInnerProduct{S}{S}}
\end{equation}
Where $(\nu_{\delta})$ has the dimensions of kinematic viscosity and represents a turbulent diffusion coefficient, $(C_{\delta})$ represents a dimensionless coefficient, and $(\sigma \propto h)$. If the spatial derivative of $(\nu_{\delta})$ is negligible then
\begin{equation}
    C_2 = \nu_{\delta} \Delta(\TildeRho)
\end{equation}
\begin{equation}
    M_2 = \nabla \cdot \tensor{T}_l = \nabla \cdot \bigg( -\frac{k^2}{3}\tensor{I} - \frac{2}{3}\nu_t \operatorname{Tr}[\widetilde{\tensor{S}}]\tensor{I} + 2\nu_t\widetilde{\tensor{S}} \bigg)
\end{equation}
\begin{equation}
    k^2 = 4 C_y \sigma^2 \FrobeniusInnerProduct{S}{S} \quad , \quad C_y=0.044
\end{equation}
\begin{equation}
    \nu_t = (C_s\sigma)^2\sqrt{2 \FrobeniusInnerProduct{S}{S}} \quad , \quad C_s=0.12
\end{equation}
Where $(Tr[])$ is the trace operator, $(C_y)$ is the Yoshizawa constant \parencite{yoshizawa1986statistical}, and $(C_s$ is the Smagorinsky constant.

The above equations, coupled with the closure models, allow the authors to replace the differential operators with SPH counterparts of WCSPH.

The authors used the  $2D$ and $3D$ TGV problems to validate the model. The authors considered Reynolds number up to $\approx \SciNot{1.2}{5}$ for the $2D$ case, and $\approx \SciNot{1.5}{3}$ for the $3D$ case. The authors also compared the proposed model against that of the DNS SPH scheme, which was developed by Mayrhofer et al. \parencite{mayrhofer2015dns} as observed in \figref{fig:DiMascio2017-fig-4} and \figref{fig:DiMascio2017-fig-9}.

\begin{figure}[H]
    \centering
    \includegraphics[scale=0.5]{Figures/research_papers/DiMascio2017-fig-4.png}
    \caption{$2D$ freely decaying turbulence $(tU/L = 2, Re = \SciNot{1.25}{5})$. Left: Energy spectrum using DNS-SPH. Right: Energy spectrum using LES- SPH. Reproduced from \cite{DiMascio2017}}
    \label{fig:DiMascio2017-fig-4}
\end{figure}
\begin{figure}[H]
    \centering
    \includegraphics[scale=0.75]{Figures/research_papers/DiMascio2017-fig-9.png}
    \caption{$3D$ homogeneous turbulence decay. Comparison between DNS-SPH and LES-SPH simulations $(tU/L = 5)$. Left: Particle resolution = $64^3$. Right: particle resolution = $128^3$. Reproduced from \cite{DiMascio2017}}
    \label{fig:DiMascio2017-fig-9}
\end{figure}

 As seen from the results of the simulations in \figref{fig:DiMascio2017-fig-4} and \figref{fig:DiMascio2017-fig-9} for the $2D$ and $3D$ case, respectively, the authors were able to claim that the model captures the typical characteristics of the flow evolution with relatively coarse particle discretisation, provided an appropriate LES modelling is used.

Building on this, Colagrossi et al. \parencite{Colagrossi2021QuasiLagrangian} introduced a small arbitrary velocity deviation $(\TildeDeltaV)$ to the fluid particle as given by
\begin{equation}
    \LagDerivative{\TildeR} = \TildeV + \TildeDeltaV 
    \label{eq:Colagrossi2021QuasiLagrangian-r-transport}
\end{equation}

This was done since the Lagrangian nature of the proposed model was proving to be an obstacle for accurate simulations of high Reynolds number problems. Therefore, they modified the model with the transport equation given in \Eqref{eq:Colagrossi2021QuasiLagrangian-r-transport} to obtain a quasi-Lagrangian LES-SPH model. The small velocity deviation to the actual Lagrangian velocity was based on the work done on particle shifting technique (PST) and tensile instability control (TIC) technique by Sun et al. \parencite{sun2018multi} in their $\delta$-SPH scheme.

The authors subsequently employed this $\delta$-LES-SPH scheme with the same set of closure models as detailed by DiMascio et al. \parencite{DiMascio2017}. They compared their proposed model with standard LES-SPH on the $2D$ and $3D$ TGV problems for Reynolds numbers up to $\SciNot{1}{6}$.

\begin{figure}[H]
    \centering
    \includegraphics[scale=0.7]{Figures/research_papers/Colagrossi2021QuasiLagrangian-fig-4.png}
    \caption{Left: Energy spectrum at $(tU/L = 3, Re = \SciNot{1}{6})$ as predicted by the $\delta$-LES-SPH model and a finite volume scheme. Symbols $k_f$ and $k_R$ indicate the wave numbers of the external forcing and the kernel radius, respectively, while $k_{\lambda}$ and $k_D$ are the wave numbers associated with the Taylor and Kolmogorov scales. Right: $k$ field at the same instant. Reproduced from \cite{Colagrossi2021QuasiLagrangian}}
    \label{fig:Colagrossi2021QuasiLagrangian-fig-4}
\end{figure}
\begin{figure}[H]
    \centering
    \includegraphics[scale=0.65]{Figures/research_papers/Colagrossi2021QuasiLagrangian-fig-8.png}
    \caption{Left: Time histories of $k$ as predicted by finite volume scheme and the $\delta$-LES-SPH model. Right: Energy spectra. Symbols $k_f$ and $k_R$ denote the wave numbers of the external forcing and the kernel radius, respectively. Reproduced from \cite{Colagrossi2021QuasiLagrangian}}
    \label{fig:Colagrossi2021QuasiLagrangian-fig-8}
\end{figure}

As seen from the results of the simulations in \figref{fig:Colagrossi2021QuasiLagrangian-fig-4} and \figref{fig:Colagrossi2021QuasiLagrangian-fig-8} for the $2D$ and $3D$ case respectively, the authors claimed that the proposed model could overcome issues of high Reynolds number flows, such as generation of spurious high-frequency noise and the onset of the tensile instability. Furthermore, comparing the proposed model’s and an FVM solve’s energy spectra with the theoretical decay rate confirmed the model’s accuracy and reliability.

The authors conclude that the $\delta$-LES-SPH model would need to be validated for flows with a Reynolds number larger than $\SciNot{1}{6}$ with experimental or numerical data. They believe that the inclusion of wall functions to deal with solid boundaries and the extension of the model to free-surface flows can further improve the model’s effectiveness. Finally, the authors believe that a higher-order approach could significantly improve the model results.

\subsection{Extension to EDAC-SPH Scheme}
\label{sec:Extension-to-EDAC-SPH-Scheme}
The Entropically Damped Artificial Compressibility (EDAC) SPH scheme, based on the work of \parencite{Clausen2013}, is an explicit SPH method devised for incompressible flow by \parencite{Ramachandran2019}.
In this scheme, the authors consider a transport equation for pressure instead of density derived from the continuity equation, as given in \Eqref{eq:Ramachandran2019-edac-p-transport}.
\begin{equation}
    \LagDerivative{P} = -c_s^2 \rho \nabla \cdot \vect{v} + \nu \nabla^2 P
    \label{eq:Ramachandran2019-edac-p-transport}
\end{equation}

Considering the pressure evolution equation above, an equation of state is longer required by the scheme.
By filtering \Eqref{eq:Ramachandran2019-edac-p-transport} with the Lagrangian LES technique \parencite{DiMascio2017}, we arrive at the result given by \Eqref{eq:edac-filtered} (Refer \appref{appendix:lagrangian-les-filtering-of-edac} for derivation).
\begin{equation}
    \LagDerivative{\TildeP} = -c_s^2 \rho \nabla \cdot \TildeV + \nu \nabla^2 \TildeP + \TildeV \cdot \nabla \TildeP - \widetilde{\vect{v}\cdot\nabla P}
    \label{eq:edac-filtered}
\end{equation}

In \Eqref{eq:edac-filtered}, we see that the term $(\widetilde{\vect{v}\cdot\nabla P})$ cannot be reduced further to a form of $(f(\TildeV, \TildeP))$. Hence, we cannot proceed with the SPH formulation for the pressure equation, implying that the EDAC scheme might not be compatible with the Lagrangian LES model.

%----------------------------------------------------------------------------------------
%	Section 4 - RANS-based $k-\epsilon$ Models
%----------------------------------------------------------------------------------------
\section[RANS-based k-epsilon Models]{RANS-based $k-\epsilon$ Models}

Shao \parencite{Shao2006} demonstrated that the two-equation $k-\epsilon$ model, an extensively studied model derived from the Reynolds-averaged Navier–Stokes (RANS) equations, can be incorporated in the truly incompressible version of SPH (ISPH). By attempting to extend RANS equations, which are hugely successful in practical fields, to a mesh-free method such as SPH, the author provides a framework to build on the wide variety of closure models available.

To discretise the RANS equations to an SPH form, the author considers the Reynolds averaged mass and momentum conservation equations as given in \Eqref{eq:Shao2006-rans-continuity-governing} and \Eqref{eq:Shao2006-rans-mom-governing} respectively. \textit{Note:} The averaged flow properties are represented without any over-line $\overline{(...)}$ hereafter.
%TODO: Use filtered terms explicitly
\begin{equation}
    \frac{1}{\rho} \LagDerivative{\rho} + \nabla \cdot \vect{v} = 0 \quad , \quad \LagDerivative{\rho} = 0 \text{ (Incompressible)}
    \label{eq:Shao2006-rans-continuity-governing}
\end{equation}
\begin{equation}
    \LagDerivative{\vect{v}} = -\frac{1}{\rho}\nabla P + \nu \Delta( \vect{v} ) + \frac{1}{\rho} \nabla \cdot \tensor{\tau} + \vect{F}
    \label{eq:Shao2006-rans-mom-governing}
\end{equation}

The stress tensor is given by \Eqref{eq:Gotoh2004-boussinesq}, while the turbulent eddy viscosity is defined as \Eqref{eq:Shao2006-turbulent-eddy-visc}.
\begin{equation}
    \nu_t = c_d \frac{k^2}{\epsilon}
    \label{eq:Shao2006-turbulent-eddy-visc}
\end{equation}

The transport equations for the turbulent kinetic energy and dissipation rate is given by \Eqref{eq:Shao2006-k-transport-eq} and \Eqref{eq:Shao2006-eps-transport-eq} respectively.
\begin{equation}
    \LagDerivative{k} = \nabla \cdot \bigg( \frac{\nu_t}{\sigma_k} \nabla k \bigg) + P_k - \epsilon
    \label{eq:Shao2006-k-transport-eq}
\end{equation}
\begin{equation}
    \LagDerivative{\epsilon} = \nabla \cdot \bigg( \frac{\nu_t}{\sigma_{\epsilon}} \nabla \epsilon \bigg) + c_{1\epsilon} \frac{\epsilon}{k} P_k - c_{2\epsilon} \frac{\epsilon^2}{k}
    \label{eq:Shao2006-eps-transport-eq}
\end{equation}
\begin{equation}
    P_k = 2\nu_t \FrobeniusInnerProduct{S}{S}
    \label{eq:Shao2006-k-production-term}
\end{equation}

Where, $(\sigma_k, \sigma_{\epsilon}, c_{1\epsilon}, c_{2\epsilon}) = (1.0, 1.3, 1.44, 1.92)$ are empirical constants dependent on the nature of the flow , and $(P_k)$ is the turbulence production rate, which satisfies the relation given by \Eqref{eq:pope2000turbulent-k-production-relation} \parencite{pope2000turbulent}.
\begin{equation}
    \frac{P_k}{\epsilon} = c_d \Bigg( \frac{\sqrt{2 \FrobeniusInnerProduct{S}{S}}}{\epsilon} \Bigg)
    \label{eq:pope2000turbulent-k-production-relation}
\end{equation}

These governing equations are solved and evolved using the same predictive-corrective time integrator as seen in the work of \parencite{Gotoh2004}, and outlined in \Eqref{eq:Gotoh2004-predict-start} - \Eqref{eq:Gotoh2004-correct-end}.

As for the SPH formulations of the governing equations, the author builds on the work of \parencite{Gotoh2004}, and uses the same discretization as defined in \Eqref{eq:Gotoh2004-summation-density} - \Eqref{eq:Gotoh2004-div-tau-sph}. However, the author uses a slightly modified version of the laminar stress term given in \Eqref{eq:Gotoh2004-vel-laplacian-sph} and redefines it as given in \Eqref{eq:Shao2006-vel-laplacian-sph}.
\begin{equation}
    \big(\nu \Delta(\vect{v}) \big)_i = \sum_j m_j \frac{2 ( \nu_i + \nu_j ) }{ \rho_i + \rho_j }\frac{\VIJ \RIJ \cdot \DWIJ}{\RtwoIJ}
    \label{eq:Shao2006-vel-laplacian-sph}
\end{equation}

The author tested the model on the problem of $2D$ wave breaking and overtopping of a sloping wall and compared the results obtained against experimental data \parencite{li2004wave} to validate the model—the computational domain of $\approx \SciNot{6}{3}$ particles.

As seen from the evolution of the water surface elevation plotted in \figref{fig:Shao2006-water-surf-elevations}, the author could ascertain that the proposed model produced better results than those of Li et al. \parencite{li2004wave}, compared to the experimental data in \figref{fig:Shao2006-water-surf-elevations}(a) and \figref{fig:Shao2006-water-surf-elevations}(b). This could be attributed to the free surfaces being accurately tracked by particles without numerical diffusion. 
In \figref{fig:Shao2006-water-surf-elevations}(c) and \figref{fig:Shao2006-water-surf-elevations}(d), despite the wave profiles being consistent with each other in phase and shape, the proposed model predicts smaller elevation levels. Li et al. use a dynamic Smagorinsky model, whereas the proposed model uses constant empirical coefficients. The author believes these coefficients derived from a quasi-steady state may behave sub-optimally in transient flow, such as the problem at hand.
\begin{figure}[H]
    \centering
    \includegraphics[scale=0.9]{Figures/research_papers/Shao2006-water-surf-elevations.png}
    \caption{Comparisons of computed water surface elevations by SPH (solid lines) with experimental ($\circ$) and numerical (dotted lines) data of Li et al. \parencite{li2004wave}. Reproduced from \cite{Shao2006}}
    \label{fig:Shao2006-water-surf-elevations}
\end{figure}

The author concludes that the $k-\epsilon$ model would require further sensitivity analysis for the turbulence model and spatial resolution for improved results, despite being reasonably accurate in tracking free surfaces.

Wang and Liu \parencite{Wang2020} build on the work of Shao \parencite{Shao2006} to further improve the ISPH $k-\epsilon$ model. They achieved this by using the modelling and computational developments which SPH has benefited from since the work of Shao.
The authors here consider the same SPH discretised equations and two-step time integrator as Shao, with two distinct modifications in the Pressure Poisson equation as given by \Eqref{eq:Wang2020-correct-pressure-implicit} instead of \Eqref{eq:Gotoh2004-correct-pressure-implicit}. They also redefined propagation equation for $(\vect{r}$ using \Eqref{eq:Wang2020-correct-end} instead of \Eqref{eq:Gotoh2004-correct-end}.
\begin{equation}
    \nabla \cdot \bigg( \frac{1}{\rho_o} \nabla P_{t+1} \bigg) = \frac{1}{\Delta t} \nabla \cdot \vect{u_*}
    \label{eq:Wang2020-correct-pressure-implicit}
\end{equation}
\begin{equation}
    \vect{r}_{t+1} = \vect{r}_* + \Delta \vect{v}_{**} \Delta t
    \label{eq:Wang2020-correct-end}
\end{equation}

The authors also provided SPH discretization for the transport equations for $(k)$ and $(\epsilon)$ using \Eqref{eq:Wang2020-k-laplacian} and \Eqref{eq:Wang2020-eps-laplacian} which requires the use of particle density $(\varphi)$ as given in \Eqref{eq:Wang2020-particle-density}.
\begin{equation}
    \varphi_i = \sum_j \WIJ
    \label{eq:Wang2020-particle-density}
\end{equation}
\begin{equation}
    \nabla \cdot \bigg( \frac{\nu_t}{\sigma_k} \nabla k \bigg)_i = - \sum_j \frac{1}{\varphi_j} \bigg( \frac{\nu_{t,i}}{\sigma_k} + \frac{\nu_{t,i}}{\sigma_k}  \bigg) \frac{k_{ij} \RIJ \DWIJ}{\RtwoIJ}
    \label{eq:Wang2020-k-laplacian}
\end{equation}
\begin{equation}
    \nabla \cdot \bigg( \frac{\nu_t}{\sigma_{\epsilon}} \nabla \epsilon \bigg)_i = - \sum_j \frac{1}{\varphi_j} \bigg( \frac{\nu_{t,i}}{\sigma_{\epsilon}} + \frac{\nu_{t,i}}{\sigma_{\epsilon}}  \bigg) \frac{\epsilon_{ij} \RIJ \DWIJ}{\RtwoIJ}
    \label{eq:Wang2020-eps-laplacian}
\end{equation}

The authors also derived an SPH formulation for the strain rate tensor defined in \Eqref{eq:Wang2020-strain-rate-tensor}, which is based on the studies done on kernel correction \parencite{bonet1999variational, khayyer2008corrected} and given by \Eqref{eq:Wang2020-vel-grad-term} and \Eqref{eq:Wang2020-corrective-matrix}. 
\begin{equation}
    \tensor{S}_i = \HalfFrac \bigg( \nabla \vect{v}_i + (\nabla \vect{v}_i)^T \bigg)
    \label{eq:Wang2020-strain-rate-tensor}
\end{equation}
\begin{equation}
    \nabla \vect{v}_i = - \sum_j \frac{1}{\varphi_j} \VIJ \otimes \tensor{L}_i \DWIJ
    \label{eq:Wang2020-vel-grad-term}
\end{equation}
\begin{equation}
    \tensor{L}_i = \Bigg(- \sum_j \frac{1}{\varphi_j} \DWIJ \otimes \RIJ \Bigg)^{-1}
    \label{eq:Wang2020-corrective-matrix}
\end{equation}

The authors validated the model against the problem of a solitary wave propagating over a bottom-mounted barrier in $2D$. Its results are shown in \figref{fig:Wang2020-wave-propogation-result}. They also considered the problem involving wave breaking on a slopping wall in $2D$, which is shown in \figref{fig:Wang2020-wave-breaking-result}.
\begin{figure}[H]
    \centering
    \includegraphics[scale=0.75]{Figures/research_papers/Wang2020-wave-propogation-result.png}
    \caption{Comparisons between experimental data (left) and numerical results (right) for turbulence intensity $(m/s)$ at $t=0.6s$ (top panels) and the corresponding vertical cross-sections (lower panels). In the lower panels, solid lines and $\circ$ represent the numerical and experimental results, respectively. Reproduced from \cite{Wang2020}}
    \label{fig:Wang2020-wave-propogation-result}
\end{figure}
\begin{figure}[H]
    \centering
    \includegraphics[scale=0.8]{Figures/research_papers/Wang2020-wave-breaking-result.png}
    \caption{Comparisons of numerical and experimental turbulent kinetic energy in x-direction and y-direction. Reproduced from \cite{Wang2020}}
    \label{fig:Wang2020-wave-breaking-result}
\end{figure}

Given the model’s accuracy, as observed from the figures, the authors conclude that the model’s capabilities have been demonstrated, especially in tracking transient-free surfaces. They also reproduce the evolution of turbulence and its intensity due to flow separation. 
However, they note that the model under-predicts the max turbulent kinetic energy and is sensitive to the initial seeding of the turbulent kinetic energy property.
The authors also state that the counteracting effects of the physical viscous dissipation and numerical dissipation, dependent on the particle resolution, would have to be appropriately balanced.
Finally, the authors conclude on the importance of boundary treatment and the need for more sophisticated boundary models to extend the model to $3D$ domains.

%TODO: Need to include Wang2022.

%----------------------------------------------------------------------------------------
%	Section 5 - LANS-based Models
%----------------------------------------------------------------------------------------
\section{LANS-based Models}
So far, having dealt with RANS and LES models of turbulence, a few fundamental drawbacks of these techniques have been brought to light.
In the case of RANS, the flow field is decomposed into
an averaged mean flow and a fluctuating field. Such a decomposition of the NS equations provides differential equations for the mean flow containing contributions from the time-varying turbulent motion. Therefore a closure model is required to capture the effects of the fluctuations on the mean flow. However, by studying only the mean flow quantities, the complete nature of the flow, notwithstanding the complex structures stemming from these fluctuations, cannot be studied, let alone appreciated.

On the face of it, LES would have made for a reasonable alternative to model turbulent flows. This is because the large scales in the flow are resolved by the standard NS equations, while the effects of smaller scales are modelled. However, this remains easier said than done since inhomogeneous and wall-bounded flows necessitate the LES filter width to vary dynamically to capture the average size of turbulent eddies.

The other widespread method of tackling turbulence involves Direct Numerical Simulation (DNS). However, since the number of degrees of freedom for a $3D$ flow grows rapidly with the Reynolds number $(\propto Re^{9/4})$, this technique effectively places an inviolable upper bound on the $Re$ for a given simulation because of computational limitations.

In order to tackle the issues above and address them reasonably, we look for possible solutions. One such solution is the Lagrangian averaging method introduced by Holm et al. \parencite{holm1998euler} and Marsden and Shkoller \parencite{Marsden2001}. In this method, unlike the averaging or filtering of the NS equations as done in RANS or LES, the Lagrangian averaging approach averages at the level of the variational principle from which the Navier–Stokes equations are derived. This procedure yields the Lagrangian-averaged Navier Stokes-$\alpha$ (LANS-$\alpha$) equations, which describe the time evolution of large eddies in turbulent flows. It can be stated that this approach is similar to that of LES.

The LANS-$\alpha$ equations, assuming incompressibility and isotropic turbulence, are derived and simplified by Mohseni et al. \parencite{Mohseni2003}. These equations are as follows:
\begin{equation}
    \nabla \cdot \vect{v} = 0
\end{equation}
\begin{equation}
    \LagDerivative{\vect{v}} = \tensor{\sigma}(\vect{v})
\end{equation}
\begin{equation}
    \tensor{\sigma}(\vect{v}) = -P\tensor{I} + 2\nu(1 - \alpha^2\Delta)\tensor{S} + 2\alpha^2\Dot{\tensor{S}}
\end{equation}
\begin{equation}
    \Dot{\tensor{S}} = \LagDerivative{\tensor{S}} + \tensor{S}\tensor{R} - \tensor{R}\tensor{S}
\end{equation}
\begin{equation}
    \tensor{R} = \HalfFrac\bigg(\nabla\vect{v}-(\nabla\vect{v})^T\bigg)
\end{equation}
Where $\alpha$ denotes the scale of rapid fluctuations in the flow map, such that for scales smaller than $\alpha$, the wave activity is filtered by a nonlinear energy redistribution. Here, $(1 - \alpha^2 \Delta)$ denotes the Helmholtz operator.

Monaghan \parencite{Monaghan2002}, building on the work of Holm \parencite{holm1998euler}, attempted to formulate an SPH version of the continuum LANS-$\alpha$ equations. His work provided one of the earliest models of turbulence in compressible flow. The author incorporated some of his earlier work on XSPH \parencite{monaghan1989problem}, in which particles are moved with a smoothed velocity, leaving the acceleration equation unchanged. The smoothed velocity denoted an average over the velocities of the neighbouring particles. This facilitated the author in writing the Lagrangian analysed by Holm in an SPH form, allowing for the subsequent derivation of the momentum equation, albeit being elaborate and highly complicated. Nevertheless, in conjunction with the continuity equation, the derived momentum equation formulated the SPH-$\alpha$ model, which essentially consisted of an SPH particle moving with the transport velocity smoothed from momentum velocity by an iterative algorithm with an additional dissipation term meant to mimic the standard LES model.

Hu and Adams \parencite{Hu2015} also devised a turbulence model for incompressible flow based on spatial filtering of the NS equations using SPH approximations termed the SPH-$\sigma$ model. The model shares similarities with the LANS-$\alpha$ model and the SPH-$\alpha$ model, differing by the additional stress term in the model and its approach in evaluating the particle transport velocity. The proposed model is also built on the authors’ previous work \parencite{hu2007incompressible}, and hence shares similar numerical techniques. The authors also validated the proposed model on $2D$ flow comprising decaying and forced turbulence cases. Their results suggested that the model could simulate incompressible turbulent flow.

Monaghan \parencite{Monaghan2011, Monaghan2017} subsequently improved on the SPH-$\alpha$ model and derived a more amenable variant of the momentum equation. Since, in this case, the model was parameterised around the smoothing parameter $(\varepsilon)$, the turbulence model was termed the SPH-$\varepsilon$ model. The linearly smoothed velocity $(\widehat{\vect{v}})$ is given by \Eqref{eq:Monaghan2017-xsph-vel-smoothing}.
\begin{equation}
    \widehat{\vect{v}}_i = \vect{v}_i - \varepsilon \sum_j \frac{m_j}{M_o} \VIJ K_{h', ij} \quad , \quad \varepsilon \in [0, 1]
    \label{eq:Monaghan2017-xsph-vel-smoothing}
\end{equation}
Where $(K)$ is a smoothing kernel, which can be different from the kernel used in SPH, its corresponding smoothing length being $(h')$. It is noted that the smoothed velocity preserves the shape of the spectrum of the unsmoothed velocity for short-length scales. However, it reduces the magnitude of the unsmoothed velocity by a factor $(1-\varepsilon)$. 

The equation of state is given by \Eqref{eq:violeau-eos}, and the momentum equation is given by \Eqref{eq:Monaghan2017-mom-governing}. The momentum equation’s third term on the right-hand side is an extra stress term determined by the smoothing. Its overall effect is to redistribute energy without dissipation.
\begin{equation}
    \LagDerivative{\vect{v}_i} = - \sum_j m_j \bigg( \frac{P_i}{\rho_i^2} + \frac{P_j}{\rho_j^2} \bigg) \DWIJ - \sum_j m_j \Pi_{ij} \DWIJ + \frac{\varepsilon}{2} \sum_j \frac{m_j}{M_o} \VtwoIJ[] \DWIJ 
    \label{eq:Monaghan2017-mom-governing}
\end{equation}
Where the viscosity term $(\Pi_{ij})$ is given by \Eqref{eq:Monaghan2017-Pi-visc-term}, in which $(\alpha)$ is a constant.
\begin{equation}
    \Pi_{ij} = - \frac{2 \alpha c_s \VIJ \cdot \RIJ}{(\rho_i + \rho_j) \RtwoIJ}
    \label{eq:Monaghan2017-Pi-visc-term}
\end{equation}

The particles are subsequently transported using \Eqref{eq:Monaghan2017-r-transport}.
\begin{equation}
    \LagDerivative{\vect{r}_i} = \widehat{\vect{v}}_i
    \label{eq:Monaghan2017-r-transport}
\end{equation}

The author validated the proposed model by simulating $2D$ flow past a cylinder moving along a Lissajous curve.
The author demonstrated the model’s capabilities to predict satisfactory results for the velocity correlation functions, energy spectrum and mixing while having particle resolution be half of that required for a DNS with a resolution the Reynolds length. The author concludes that the model’s effectiveness for higher Reynolds numbers and other boundary conditions, such as free surfaces, will need to be studied further.

%----------------------------------------------------------------------------------------
%	Section 6 - Miscellaneous Models
%----------------------------------------------------------------------------------------
\section{Miscellaneous Models}
Liu et al. \parencite{Liu2019} approach the problem of turbulence modelling from the perspective of computer $3D$ visualisation to visualise flows with realistic features. The authors’ viscosity-based vorticity correction model offers more advantages to the computer visual-effects audience. However, the model would still be a worthwhile investigation.

The authors define a vorticity field $(\Vorticity)$ in the system as given in \Eqref{eq:Liu2019-vorticity}.
\begin{equation}
    \Vorticity_i = \nabla \times \vect{v}_i = - \frac{1}{\rho} \sum_j m_j \VIJ \times \DWIJ
    \label{eq:Liu2019-vorticity}
\end{equation}

The contribution to the SPH particle’s momentum velocity from the viscous term is calculated as given in \Eqref{eq:Liu2019-visc-vel-contribution}.
\begin{equation}
    \vect{v}_{\nu, i} = \vect{v}_i + \vect{a}_i^{(visc)}\Delta t
    \label{eq:Liu2019-visc-vel-contribution}
\end{equation}
If the magnitude of viscous velocity $(\vect{v}_{\nu, i})$ is greater than $(\vect{v}_i)$ for a given particle, then its vorticity field is updated as given in \Eqref{eq:Liu2019-vorticity-update}.
\begin{equation}
    \delta \Vorticity_i = (\alpha \sqrt{R_i}) \Vorticity_i
    \label{eq:Liu2019-vorticity-update}
\end{equation}
\begin{equation}
    R_i = \frac{\delta E_i}{E_i} = \frac{\vect{v}_{\nu, i}^2 - \vect{v}_i^2}{\vect{v}_i^2}
    \label{eg:Liu2019-R}
\end{equation}

A stream function $(\Psi)$ is subsequently defined as given in \Eqref{eq:Liu2019-stream-func}.
\begin{equation}
    \Psi_i = \sum_j \frac{\delta \Vorticity_j \Vol_j}{4\pi \RtwoIJ[]}
    \label{eq:Liu2019-stream-func}
\end{equation}

The velocity correction $(\delta \vect{v})$ for the particle is subsequently defined by \Eqref{eq:Liu2019-vel-update}.
\begin{equation}
    \delta \vect{v}_i = \nabla \times \Psi_i
    \label{eq:Liu2019-vel-update}
\end{equation}

Obeidat and Bordas \parencite{Obeidat2018}, tackle the problem of turbulence by devising a hybrid remeshed method (hrSPH)
for the simulation of three-dimensional turbulent flows. 
The authors essentially use an Eulerian mesh with Lagrangian particles to gain advantages of both schemes. The governing equations are evaluated using the properties at the mesh nodes, which are interpolated from the particles. After evaluating the system of equations, the new properties on the mesh nodes are used to update the properties of the particles through mesh-to-particle interpolation. The authors consider the turbulent sub-grid stresses using the Smagorinsky model.

The proposed model remeshes the particles when the distribution is not uniform. The authors validated the model against the $2D$ and $3D$ TGV problems, thin double shear layer problems, and $3D$ isotropic turbulence obtained from DNS. The model predicts the energy spectra and energy dissipation reasonably well. Despite the model’s accuracy, not much information can be gathered to help solely improve the state of turbulence modelling in SPH.