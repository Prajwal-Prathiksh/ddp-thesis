\chapter{Kutta-Joukowski Theorem} % Main chapter title
\label{Chapter3}

\section{Inviscid Flow}
W. M. Kutta in 1902 and independently by N. Joukowski in 1906, derived the expression for lift per unit depth as generalized it as follows \parencite{White2018-ai}:
\emph{According to inviscid theory, the lift per unit depth of any cylinder of any shape immersed in a uniform stream equals, where is the total net circulation contained within the body shape. The direction of the lift is $90^{\circ}$ from the stream direction, rotating opposite to the circulation.}
\begin{equation}
	\frac{L}{b} = - \rho V_{\infty} \Gamma
	\label{eq:classic-kjt-lift}
\end{equation}
As for drag, D'Alembert published the result \parencite{White2018-ai}:
\emph{According to inviscid theory, the drag of any body of any shape immersed in a uniform stream is identically zero.}
\begin{equation}
	D = 0 \text{ for cylinder with circulation}
	\label{eq:classic-kjt-drag}
\end{equation}
Liu et al. \parencite{Liu2015}, by running a RANS simulation on a typical airfoil, were able to show that the Kutta-Joukowski Theorem (KJT) is a good approximation for real viscous flow in typical aerodynamic applications. It should however be noted that this classic formulation of KJT detailed in eq. \ref{eq:classic-kjt-lift} and \ref{eq:classic-kjt-drag} is applicable for a steady potential flow around a single airfoil (or cylinder), wherein the lift is related to the circulation of the bound vortex.

\section{Generalized Kutta-Joukowski Theorem}
Bai et al. \parencite{Bai2014}, generalized the classic KJT in order to obtain force formulas for single and multibody flows with multiple free and bound vortices and vortex production. They did so by starting with a momentum balance analysis that was based on an appropriately designed control volumes different for lift and drag. They also ensured that the influence of the vortex production on the forces was dealt in a simple, explicit and frame-independent manner.
The crux of their work is as follows:
\begin{itemize}
	\item Conservation of the total circulation:
	\begin{equation}
		\sum_i \frac{d \Gamma_i}{d t} = 0
		\label{eq:gkjt-conservation-of-circulation}
	\end{equation}
	Where, ($i$) refers to any singularity independent of whether it is inside or outside the body, and ($\Gamma_i$) is the circulation of the $i^{th}$ vortex
	\item Lift:
	\begin{equation}
		L = - \rho \sum_i \bigg( V_{\infty} \Gamma_i - \frac{d(\Gamma_i x_i)}{dt} \bigg) + \rho \sum_i \frac{d(m_i y_i)}{dt} + L_{add}
		\label{eq:gkjt-lift}
	\end{equation}
	Where, ($x_i, y_i$) refers to the position of a vortex, source or sink at a given instant of time, and ($m_i$) refers to the strength of the source or sink.
	\item Drag:
	\begin{equation}
		D = - \rho \sum_i \frac{d(\Gamma_i y_i)}{dt} + \rho \sum_i \frac{d(m_i x_i)}{dt} + D_{add}
		\label{eq:gkjt-drag}
	\end{equation}
	
	Here ($L_{add}, D_{add}$) are the forces due to body acceleration and rotation, and note that $\sum_i$ is always performed over all the inner and outer singularities.	
\end{itemize}
If we assume that $\Gamma_i$ is a constant and that the body in question is fixed, the force eqs. \ref{eq:gkjt-lift} and \ref{eq:gkjt-drag} can be simplified to:
\begin{equation}
	L = - \rho \sum_i \bigg( V_{\infty} - \frac{d x_i}{dt} \bigg) \Gamma_i
	\label{eq:gkjt-simple-lift}
\end{equation}
\begin{equation}
	D = - \rho \sum_i \frac{d y_i}{dt} \Gamma_i
	\label{eq:gkjt-simple-drag}
\end{equation}
Where ($dx_i/dt, dy_i/dt$) is the velocity of the point vortex.
Here we can see that in the special case when the outside vortices, i.e., the starting vortices, move at the free-stream velocity ($V_{\infty}$) and the internal vortices are fixed, as in the case of steady flow, we recover the classical KJT from eqs. \ref{eq:gkjt-simple-lift} and \ref{eq:gkjt-simple-drag}:
\begin{equation}
	L = - \rho V_{\infty} \Gamma_b
\end{equation}
\begin{equation}
	D = 0
\end{equation}
\begin{equation}
	\Gamma_b = \sum_{i \in \text{internal}} \Gamma_i
\end{equation}
Where, ($\sum_{i \in \text{internal}} \Gamma_i$) refers to the summation over all the vortices inside the body.
Bai et al. further rewrote eqs. \ref{eq:gkjt-lift} and \ref{eq:gkjt-drag} so as to relate the forces to the induced velocities as opposed to the actual velocity of a particular singularity. This allowed them to derive further conclusions, which included:
\begin{itemize}
	\item Interaction between free singularities does not contribute to the forces
	\item The induced velocity effect is due to the interaction between free and inner singularities
	\item The force formulas expressed in a physically arranged and explicit form allows for the identification of roles of different force contributions and force decomposition
	\item An optimization algorithm can be designed to optimize the arrangement of outside vortices and bodies for force enhancement or reduction since the relationship between the forces and singularities is explicit
\end{itemize}

\section{Von Kármán vortex street}
Kármán vortex street is a repeating pattern of swirling vortices which is caused by vortex shedding, and is responsible for the unsteady separation of flow of a fluid around blunt bodies. It is one of most extensively investigated structure in fluid mechanics.
When Kármán analyzed the stability of these vortex street based on the model of two infinite rows of point vortices (as shown in fig. \ref{fig:karman-vortex-street}), he found that for all values of the ratio ($k = h/l$), where ($h$) is the distance between the rows and ($l$) is the distance between the vortices in each row, there is a set of infinitesimal harmonic perturbations with an exponentially growing amplitude. The only exception for a stable vortex street is if it follows the criteria \parencite{Karmán1912}:
\begin{equation}
	k = \frac{h}{l} = \frac{1}{\pi} \cosh^{-1}(\sqrt{2}) \approx 0.281
	\label{eq:karman-street-stability-condition}
\end{equation}
Though the condition in eq. \ref{eq:karman-street-stability-condition} is very well satisfied in experiments, theoretically, the precise nature of this unique stable solution, is still under discussion \parencite{jimenez1987linear}.

Furthermore, Saffman \parencite{saffman1982stability} studied a vortex street model which consisted of two infinite rows of spread vortices. He was able to show that for vortices of finite size, there exists a finite range of ratio ($k$) for which the street is stable to infinitesimal perturbations. He subsequently concluded that further research should be based on direct numerical calculations.
\begin{figure}[H]
	\centering
	\includegraphics[width=13cm]{Figures/Karman-Street}
	\caption{Kármán vortex street}
	\label{fig:karman-vortex-street}
\end{figure}

\subsection{Reverse Von Kármán vortex street} 
\begin{figure}[H]
	\centering
	\includegraphics[width=11cm]{Figures/comparison-of-wakes}
	\caption{Comparison of wakes formed behind a stationary cylinder (A) and those produced by a swimming fish (B) - Reproduced from \parencite{WillisJay2013}}
	\label{fig:comparison-of-wakes}
\end{figure}
A biological swimmer, or analogously a heaving airfoil, through its oscillatory motion in the fluid can propel itself through the formation of vortices in the wake pattern acting like a jet to provide thrust. This wake pattern is different from the one produced when there is flow past a blunt-body as depicted in fig. \ref{fig:comparison-of-wakes}. This wake pattern is termed as the reverse Von-Kármán vortex street where the vortices created by the pitching and heaving foil are shed opposite one another in rotational direction and with an outward direction respective to the foil’s top and bottom surfaces.
\begin{figure}[H]
	\centering
	\includegraphics[width=12cm]{Figures/drag-and-thrust-wake-profiles}
	\caption{Reverse Von Kármán vortex street created in wake of a flapping foil - Reproduced from \parencite{Elles2016AnalysisOF}}
	\label{fig:drag-and-thrust-wake-profiles}
\end{figure}
The reversed shedding pattern creates a thrust channel behind the foil body rather than a mean velocity deficit, if the direction of vortex rotation is opposite as seen in fig. \ref{fig:drag-and-thrust-wake-profiles}. Therefore, a heaving foil allows for the transformation of drag into thrust \parencite{Elles2016AnalysisOF}.

This phenomenon was originally studied by Knoller \parencite{knoller1909gesetzedes} and Betz \parencite{betz1912beitrag}, who identified that a heaving wing creates an effective angle of attack, that results in a normal force vector with both lift and thrust components. Subsequently, Katzmayr \parencite{katzmayr1922effect} provided the experimental verification of the Knoller–Betz effect, by placing a stationary airfoil in a sinusoidally oscillating wind stream and measured an average thrust.
The reason why one would be interested in studying these reverse  Von Kármán vortex streets, is that these methods of aerodynamic propulsion observed in nature has a higher propulsive efficiency than that of a simplified propeller model because of the disadvantageous trailing vortex system generated by the propeller as noted by Kuchemann et al. \parencite{kuchemann1953aerodynamic}.
In further experimental investigations of this phenomenon involving studies of airfoils performing angular accelerations and vertical translation, it was observed that with an increase in the frequency and/or amplitude of oscillations, a deviation of the wake from the axis of symmetry was observed \parencite{Jones1998}, as depicted in fig. \ref{fig:vortex-wake-behind-oscillating-airfoil}.
\begin{figure}[H]
	\centering
	\includegraphics[width=8cm]{Figures/vortex-wake-behind-oscillating-airfoil}
	\caption{A vortex wake behind an airfoil performing translational vertical oscillations - Reproduced from \parencite{Jones1998}. Top panel: Calculations by discrete vortex method. Bottom panel: Visualization of flow in the experiment}
	\label{fig:vortex-wake-behind-oscillating-airfoil}
\end{figure}

\subsection{Stability of reverse von Kármán vortex street}
Dynnikova et al. \parencite{Dynnikova2021} showed that the existence of a wave with an exponentially growing amplitude means instability of the infinite street of reverse vortices, by building on the stability analysis done on the classic von Kármán vortex streets by Kármán himself \parencite{Karmán1912}.
\begin{figure}[H]
	\centering
	\includegraphics[width=8cm]{Figures/semi-inifinite-rkvs-stability}
	\caption{Region of instability of a semi-infinite reverse von Kármán vortex street - Reproduced from \parencite{Dynnikova2021}}
	\label{fig:semi-inifinite-rkvs-stability}
\end{figure}
As depicted in fig. \ref{fig:semi-inifinite-rkvs-stability}, Dynnikova et al. were able to show that for small values ($\Gamma / (l V_{\infty})$) and ($k$), the reverse street is stable.

In fig. \ref{fig:rvks-various-stouhal-number} for the flow depicted at ($St = 0.6$), the values of ($\Gamma / (l V_{\infty})$) and ($k$) are approximately $0.74$ and $0.06$, which lies in the stability region. As the Strouhal number increases, the values of  ($\Gamma / (l V_{\infty})$) and ($k$) increases as well, which leads to the transition region and subsequently the instability region, as well as to an increases in the ratio of ($l/\lambda$) in the region. Here, ($\lambda$) represents the wavelength of the sinusoidal perturbation.
\begin{figure}
	\centering
	\includegraphics[width=9cm]{Figures/rvks-various-strouhal-number}
	\caption{Vortex wakes behind an airfoil performing angular oscillations ($(Re) = 1000$). Gray and black colors correspond to the vortex particles with positive (counterclockwise) and negative (clockwise) circulation. \emph{($Sh$): Strouhal number} - Reproduced from \parencite{Dynnikova2021}}
	\label{fig:rvks-various-stouhal-number}
\end{figure}

\section{Simulation Framework}
In order to gain further insights on the evolution of various configurations of vortices with and without a free-stream, a \texttt{Vortex} class and a \texttt{VortexShedder} class, in addition to an accompanying \texttt{VortexPropagator} were developed on Python.
The \texttt{Vortex} is initialised with:
\begin{itemize}
	\item Type of vortex: \texttt{"free"} or \texttt{"bound"}
	\item $\Gamma$: Strength of the vortex
	\item $r$: Initial location of the vortex
\end{itemize}
Once the vortices \textit{(both free and bound vortices combined)} are initialised, the \texttt{VortexPropagator} is called upon to evolve the vortices in time, by calculating the induced velocity on a specific vortex caused by all of the other vortices. With the induced velocity calculated, the position of said vortex is propagated using the calculated induced velocity (and the free-stream velocity, if any) using an appropriate time-integrator. For now, the Euler integrator is used for its simpler implementation as well as the fact that debugging is relatively easier with a simpler time-integrator. However, for the final simulation framework, a more stable and accurate time-integrator will be used.

The \texttt{VortexShedder} class is used to set-up a train of ($N$) vortices which are shed at specified angles with respect to the center of the reference body. The ($V_{\infty}$) and ($St$) are also specified for the instance to calculate the time-interval between two successive vortex-shedding events.

In order to verify the code implementation, the following test-cases were run:
\begin{figure}[H]
	\centering
	\includegraphics[width=13cm]{Figures/2V_2S_0U}
	\caption{2 Positive Vortices ($V_{\infty} = 0$)}
	\label{fig:2V_2S_0U}
\end{figure}
\begin{figure}[H]
	\centering
	\includegraphics[width=13cm]{Figures/2V_2S_1U}
	\caption{2 Positive Vortices ($V_{\infty} = 1$)}
	\label{fig:2V_2S_1U}
\end{figure}
\begin{figure}[H]
	\centering
	\includegraphics[width=13cm]{Figures/2V_1S_0U}
	\caption{1 Positive and 1 Negative Vortex ($V_{\infty} = 0$)}
	\label{fig:2V_1S_0U}
\end{figure}
\begin{figure}[H]
	\centering
	\includegraphics[width=13cm]{Figures/2V_1S_1U}
	\caption{1 Positive and 1 Negative Vortex ($V_{\infty} = 1$)}
	\label{fig:2V_1S_1U}
\end{figure}
\begin{figure}[H]
	\centering
	\includegraphics[width=13cm]{Figures/3V_3S_0U}
	\caption{3 Positive Vortices ($V_{\infty} = 0$)}
	\label{fig:3V_3S_0U}
\end{figure}
\begin{figure}[H]
	\centering
	\includegraphics[width=13cm]{Figures/3V_2S_0U}
	\caption{2 Positive and 1 Negative Vortex ($V_{\infty} = 0$)}
	\label{fig:3V_2S_0U}
\end{figure}
As expected in the two-vortex cases where ($V_{\infty} = 0$) (refer fig. \ref{fig:2V_2S_0U}), we observe the vortices start rotating each other about a common axis, which still remains true when ($V_{\infty} = 1$) as well, except that the superimposed horizontal translation if also visible (refer fig. \ref{fig:2V_2S_1U}.
In the case of a vortex dipole, we observe that the vortices pair up, and move translate towards infinity without any complicated patterns interlaced (refer figs. \ref{fig:2V_1S_0U} and \ref{fig:2V_1S_1U}).
For the three-vortex problems, we observe the simple circular motion in the symmetric and stable configuration of three positive vortices with ($V_{\infty} = 0$) (refer fig. \ref{fig:3V_3S_0U}). We also begin to see the advent of chaotic motion in the trajectories of the vortices as their initial configuration loses symmetry (refer fig. \ref{fig:3V_2S_0U}).
