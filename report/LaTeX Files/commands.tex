% ---------- CUSTOM COMMANDS ----------

% Option to move href url to footnote for printing
\newcommand{\printurl}[2]{ % Options below are mutually exclusive!
   \href{#1}{#2} % Uncomment for standard href
  % #2\footnote{#1} % Uncomment for print mode, URL in footnote
}

% Better highlighting for inline code
\newcommand{\linecode}[1]{%
  \colorbox{lstbg}{\textcolor{lstStr}{\textbf{\texttt{#1}}}}%
}

% Outputs 'Chapter X' (or translation) as a clickable link
\newcommand{\chapref}[1]{%
  \hyperref[#1]{Chapter \ref{#1}}%
}

% Outputs 'Section X' as a clickable link
\newcommand{\secref}[1]{%
  \hyperref[#1]{Sec. \ref{#1}}%
}

% Outputs 'Figure X' as a clickable link
\newcommand{\figref}[1]{%
  \hyperref[#1]{Fig. \ref{#1}}%
}

% Outputs 'Eq X' as a clickable link
\newcommand{\Eqref}[1]{%
  \hyperref[#1]{Eq. \ref{#1}}%
}

% Outputs 'Appendix X' as a clickable link
\newcommand{\appref}[1]{%
  \hyperref[#1]{Appendix \ref{#1}}%
}

%% Math Commands
\newcommand{\bb}[1]{\mathbb{#1}}

\newcommand{\LagDerivative}[2][t]{\frac{\operatorname{D} #2}{\operatorname{D} #1}}

\newcommand{\PartialDerivative}[2][t]{\frac{\partial #2}{\partial #1}}

\newcommand{\vect}[1]{\mathbf{#1}}

\newcommand{\tensor}[1]{\underline{\bm{#1}}}

\newcommand{\WIJ}{W_{h, ij}}

\newcommand{\DWIJ}{\nabla_{i} W_{h, ij}}

\newcommand{\VIJ}{\vect{v}_{ij}}

\newcommand{\RIJ}{\vect{r}_{ij}}

\newcommand{\RtwoIJ}[1][2]{|\vect{r}_{ij}|^{#1}}

\newcommand{\MachineEpsilon}{\bm{\xi}}

\newcommand{\EncAngBrk}[2][]{#1\langle #2 #1\rangle}

\newcommand{\RAProp}[1]{\EncAngBrk{#1}}

\newcommand{\FrobeniusInnerProduct}[2]{\EncAngBrk{\tensor{#1}, \tensor{#2}}}

\newcommand{\FrobeniusNorm}[1]{||\tensor{#1}||_F}

\newcommand{\BasisVect}[1][i]{\hat{\vect{e}}_{#1}}

\newcommand{\HalfFrac}{\frac{1}{2}}

\newcommand{\Vol}[1][V]{\mathcal{#1}}

\newcommand{\Vorticity}{\bm{\omega}}

\newcommand{\Boundary}{\bm{\Omega}}

\newcommand{\WaveNumber}{\mathrm{k}}

\newcommand{\WaveNumberVector}{\vect{k}}

\newcommand{\TildeR}{\widetilde{\vect{r}}}

\newcommand{\TildeV}{\widetilde{\vect{v}}}

\newcommand{\TildeDeltaV}{\delta\TildeV}

\newcommand{\TildeRho}{\widetilde{\rho}}

\newcommand{\TildeP}{\widetilde{P}}

\newcommand{\IntRThreeAndT}{\int_{\mathbf{R}^3}
\int_{-\infty}^{\infty}}

\newcommand{\SciNot}[3][10]{#2 \times {#1}^{#3}}

\newcommand{\IntD}{d \tau d V_y}

\newcommand{\TilePArgRp}{\TildeP(\TildeR_p, t)}

\newcommand{\PhiRY}{\phi (\TildeR_p(t)-\vect{y}, t-\tau)}

\newcommand{\PY}{P(\vect{y}, \tau)}

\newcommand{\VY}{\vect{v}(\vect{y}, \tau)}

\newcommand{\TildeVArgRp}{\TildeV (\TildeR_p, t)}
