\chapter{Vortex Panel Method} % Main chapter title
\label{Chapter4}

\section{Introduction}
The Vortex Panel Method is a quite simple yet effective computational tool for researchers studying the aerodynamics of airfoil sections. The vortex panel method is a method for computing ideal flows, i.e., flows in which the effects of viscosity and compressibility are negligible. Such an ``incompressible'' flow, proves to be an effective assumption to work in owing to its simplicity, and the ability to derive considerable insights and physical intuitions for the flow field, and its evolution into either a steady or unsteady state.
Research based on this method has been conducted for large parts of flows past ships, submarines, cars and light aircraft where the flow is closely ideal and incompressible \parencite{Bertin1998-wg}.

We know that the incompressible fluid flows are essentially solutions to Laplace’s equation. Since this differential equation is linear, we can add together any number of solutions - `superpositioning', in order to generate a whole solution which represents the flow of a complex problem. This is essentially what is done in the standard vortex panel method, where elementary flow artefacts such as point vortices are employed to satisfy the required boundary conditions of the problem, for the analysis of airfoil flows.

The vortex panel method \parencite{Kuethe1987-qq} models the flow past an airfoil as the net effect of a uniform flow and a series of vortex panels, which are arranged to form a closed polygon with a shape that approximates, as nearly as possible, the actual curved shape of the airfoil (refer fig. \ref{fig:VPM-airfoil-geometry}). 

\begin{figure}[H]
	\centering
	\includegraphics[width=10cm]{Figures/VPM-airfoil-geometry.png}
	\caption{Panelized Airfoil Geometry \textit{(Anticlockwise ordering of panels, red lines: normal vectors for each panel)}}
	\label{fig:VPM-airfoil-geometry}
\end{figure}
\begin{figure}[H]
	\centering
	\includegraphics[width=10cm]{Figures/VPM-airfoil-control-points.png}
	\caption{Control points in the Airfoil Geometry}
	\label{fig:VPM-airfoil-control-point}
\end{figure}

In general, the strength of a panel can vary along its length in an arbitrary way. However, for the scope of this project, the vortex panel method was implemented by assuming that the strength of each panel is constant. This means that, after defining $N$ panels around an airfoil, there are $N$ strengths to specify. The challenge here is to choose these strengths such that the flow past the panels is a realistic representation of the flow over an airfoil. To do this we invoke the \textit{'no-penetration’} condition - the condition that the flow cannot pass through the surface of the airfoil or, equivalently, that the component of velocity perpendicular to the airfoil surface is zero. 

We apply this condition approximately by writing equations for the velocity generated by all the panels, plus the free stream, at the central point of each panel (called the control points, refer fig. \ref{fig:VPM-airfoil-control-point}). We then set the component of this velocity perpendicular to the panel equal to zero.  Since we have $N$ control points this gives us $N$ equations for the $N$ strengths. However, since the flow also has to satisfy the Kutta Condition \parencite{Clancy1986-sd, Kuethe1987-qq}, this provides us with an additional constraint, therefore rendering the system of equations over-determined. The Kutta condition essentially encapsulates the observation that the flow cannot go around the trailing edge, but must leave the airfoil there. This is a consequence of the viscous effects, which are fundamentally absent from the calculation. Therefore, for the Kutta condition to be satisfied the strengths of the vortex panels must be equal and opposite where they meet at the trailing-edge joint. As detailed in a later section, different implementations of the Kutta condition were considered and studied, before finalizing a standard formulation for the rest of the study.

The vortex panel method was chosen for the effect of vortex shedding from a heaving airfoil, since it proved to be a simple and fast method of computation, allowing for rapid development and testing of the flow models, as opposed to complex, high-fidelity CFD solvers.

%The basic equations for the VPM are as follows:
%TODO-\eq all equations for VPM

\section{Rankine Vortex}
Vortices in nature, when modelled as an irrotational or potential vortex, suffer from the fact that the velocity becomes infinite at the vortex centre, thus leading to a singularity at its centre. However, in reality, it has been observed that very close to the origin, the motion of the vortex resembles that of a solid body rotation. This is exactly what the Rankine vortex model assumes - a solid-body rotation inside a cylinder of radius $a$ and a potential vortex outside the cylinder, therefore providing a simple mathematical model of a vortex in a viscous fluid \parencite{Acheson1990-cp}. The radius $a$ is referred to as the vortex-core radius in literature. In this implementation, it is termed as the \textit{desingularize-radius}.The velocity components of such a vortex are as follows:
\begin{equation}
	v_r=0, v_{\theta}(r)=\frac{\Gamma}{2\pi}\begin{cases}
			r/a^2, & r\leq a\\
            1/r, & r>a
		 \end{cases}
	\label{eq:rankine-velocity}
\end{equation}

Modelling the shed vortices, also termed the free external vortices as Rankine vortices seemed beneficial. This is because when shed vortices were modelled as point vortices and ended up coming close to each other, they would shoot off towards infinity at a significant speed and did not behave as expected from experimental data \parencite{Millikan2018-oc, Massey1998-zr}.
The velocity distribution in a Rankine vortex is plotted in fig. \ref{fig:rankine-vortex-vel-field}.

\begin{figure}[H]
	\centering
	\includegraphics[width=12cm]{Figures/rankine-vortex-vel-field.png}
	\caption{Rankine Vortex Velocity Field}
	\label{fig:rankine-vortex-vel-field}
\end{figure}


\section{Growth of Trailing-Edge Vortices \& Their Detachment}
Once the modelling of the free external vortices is decided upon, we now come to the task of modelling the growth of these vortices, and their eventual detachment from the airfoil, leading to the generation of a new vortex which is attached to the airfoil.
Tackling the problem of identifying when an attached vortex will detach is made easier, stemming from the fact that if the instantaneous bound vorticity over the airfoil (calculated by summing over all of the panel vortices on the airfoil after having calculated it from VPM) is lower than what it was at the immediate previous time-instant, then the attached vortex cannot be fed any further from the bound vorticity of the airfoil, and hence the vortex sheet stemming out from the trailing edge of the airfoil will break, allowing for the vortex to now travel downstream without further growth.
As the instant this happens, another vortex will be created at the specified starting distance from the trailing edge of the airfoil, and will continue to grow in strength until the aforementioned detachment condition is satisfied.

\begin{figure}[H]
	\centering
	\includegraphics[width=10cm]{Figures/shed-vortex-schematic.png}
	\caption{A schematic of shed vortices - Reproduced from \parencite{nasa-shed-vortex} }
	\label{fig:shed-vortex-schematic}
\end{figure}

Now in order to model the growth of the attached vortex itself, initially, a naive implementation based on the Helmholtz's third theorem, essentially stating the conservation of vorticity was considered, wherein at each time instant, the strength of the attached vortex was negative of the sum of all of the panel vortices.
\begin{equation}
    \Gamma_{t_i} = \sum_j \gamma_{j, t_i}, j \in \text{panel vortices}
\end{equation}
However, this did not allow for convergence in the solutions, particularly the magnitude and polarity of these free vortices. Furthermore, with time, the magnitude of these vortices would diverge very quickly.
Therefore, the method employed by Xia et. al. \parencite{Xia2017}, was considered, wherein the rate of growth of the attached vortex is specified in terms of the velocity of the trailing edge panels as specified:
\begin{equation}
    \frac{d \Gamma_{t_i}}{d t} = \frac{1}{2} (V_{t, 1}^2 - V_{t,n}^2)
\end{equation}
Where $V_{t, 1}$ refers to the tangential velocity of the first panel on the bottom side of the trailing edge, and $V_{t, n}$ refers to the tangential velocity of the last panel on the upper side of the trailing edge.
With this implementation, we could observe convergence through the bounded and periodic behaviour in the magnitude of the vortex strength, and the appropriate polarity for the free vortex strengths given the nature of the flow. Hence this implementation was finalized for the study.


\section{Kutta Condition - Implementation}
The Kutta condition, representing an additional constraint in the model, is required so as to encapsulate the observation that the flow cannot go around the trailing edge, but must leave the airfoil there. Therefore the following five different variants were considered capable of modelling the Kutta condition adequate:
\begin{itemize}
    \item Implementation `A'
    \begin{equation}
        \gamma_1 + \gamma_n = 0
    \end{equation}
    \item Implementation `B'
    \begin{equation}
        \sum_{i=1}^{n-1} - K_{Ni} \gamma_i = b_n, \gamma_1 + \gamma_n = 0
    \end{equation}
    \item Implementation `C'
    \begin{equation}
        \sum_{i=1}^{n-1} - K_{Ni} \gamma_i = b_n, \gamma_1 = \gamma_n = 0
    \end{equation}
    \item Implementation `D'
    \begin{equation}
        V_{n, 1} = V_{n, n} = 0
    \end{equation}
    \item Implementation `E'
    \begin{equation}
        V_{t, 1} = V_{t, n} = 0
    \end{equation}
\end{itemize}
From multiple testing, it was observed that with the exception of `D', the other implementations gave the same solution within a tolerance of $10^-5$ with the verification test cases. Therefore implementation \textbf{A} was finalized for the study.

\section{System Solver}
%TODO-In order to solve the system of equations, and calculate the strength of the vortices which satisfies TODO-EQNUM, while taking into account the Kutta condition, the initial approach was to invert the A matrix and solve for the gamma matrix.
In order to solve the linear system of equations for the vortex panel method, while ensuring the Kutta condition is satisfied, the initial approach was to invert the matrix in order to find the strengths of all of the panel vortices.
However, this presented a problem, wherein with time, the panel vortex strengths - due to the presence of the attached and external vortices started diverging in an unbounded fashion based on the polarity of the vortex strengths at the previous time instant, i.e., $\gamma$ values which were negative became further and further negative and vice-versa for the positive $\gamma$ values.
A second approach involving the Least-Squares approximation was considered for solving the vortex strengths. However, this yielded in the same behaviour. Hence to overcome the issue of diverging panel vortex strengths, a Bounded Least-Squares approximation was considered, where the bound for each $\gamma_i$ was formulated as follows:
\begin{equation}
    \gamma_{t+\Delta t, i} \in \begin{cases}
			(-\gamma_{max}, 0), & \gamma_{t} < 0\\
            (0, \gamma_{max}), & \gamma_{t} \geq 0
		 \end{cases}
\end{equation}
\begin{equation}
    \gamma_{max} \propto \alpha_{eff} V_{\infty}^2
\end{equation}
Where $\alpha_{eff}$ is the effective angle of attack, which takes into account the relative velocity of the airfoil along the $y$-axis as well as the fixed angle of attack.
The reasoning behind this formulation stems from the fact the maximum bound vorticity over an airfoil would be proportional to the instantaneous lift it can generate (Blasius Theorem), and therefore the corresponding upper bound for the absolute magnitude of the panel vortex strength would also be proportional to this quantity.
It was also observed that using an unbounded solver, albeit a matrix inversion or the least-squares approach, the vortices would travel through the airfoil itself and move upstream, which was not the case using a bounded solver.

\begin{figure}[H]
	\centering
	\includegraphics[width=13cm]{Figures/solver-bounded-unbounded.png}
	\caption{Least-squares with bounds \textit{(Above)}; Least-squares \textit{(Below)}}
	\label{fig:solver-bounded-unbounded}
\end{figure}

\section{Code - Implementation}
The simulation was written using the principles of Object-Oriented-Programming (OOP) in Python, for a clear understanding and implementation of code. Furthermore, in order to speed up computationally heavy, or highly repetitive tasks - such as calculating the streamlines, or calculating the induced velocity at a point due to multiple vortices, stand-alone functions were written making use of \texttt{numpy} and JIT compilers using the \texttt{numba} package, in order to increase the computational speed compared to that of a standard Python.
The entire simulation was built up using three main classes as outlined below:
\subsection{\texttt{Vortex}}
A class to instantiate a point or Rankine vortex with properties such as its location, strength and type of vortex i.e., `bound' or `free'.

\subsection{\texttt{VortexSystem}}
A class designed to instantiate a model of a dynamic system of free and attached vortices with properties such as a list of free vortices, a list of attached vortices (which can be free or bound themselves), and desingularize-radius for the system.

\subsection{\texttt{MVPM Solver}}
A class designed to set up and run the entire simulation built on top of the \textit{`Modified-VPM'} as outlined so far.

\SetKwComment{Comment}{/* }{ */}
\subsubsection{\texttt{Initializer}}
\begin{algorithm}[H]
    \caption{Initializer}
    $\text{control-points}, \gamma, V_t, C_p \gets \text{VPM-Solver}$;
    \\ Update initial panel properties;
    \\ Compute force coefficients;
    \\ Save data;
    \\ Set-up vortex system;
\end{algorithm}

\subsubsection{\texttt{Propagator}}
\begin{algorithm}[H]
    \caption{Propagator}
    \While{$t < T_{end}$}{
        Update position of airfoil and attached vortex;
        \\ Update the positions of free vortices;
        \\ $\text{control-points}, \gamma, V_t, C_p \gets \text{VPM-Solver}$;  \Comment*[r]{Solve for new panel properties}
        \\ Compute force coefficients;
        \\ Update the strength of the attached vortex;
        \\ Check for vortex shedding criteria;
        \\ Dump simulation data;
        \\ $t \gets t + \Delta t$;
    }
\end{algorithm}


\section{Code - Verification}
In order to verify the correctness of the code, against simple verification test cases for which results exist, the following cases where an airfoil in free-stream which does not exhibit any heaving motion was considered. The results were compared with that from XFOIL \parencite{drela1989xfoil}. They results are detailed below:
\subsection{Symmetrical Airfoil (NACA0012, $\alpha=0^{\circ}$)}
\begin{figure}[H]
	\centering
	\includegraphics[width=13cm]{Figures/0012-0/airfoil-streamlines.png}
	\caption{Streamlines - NACA0012, $\alpha=0^{\circ}$}
	\label{fig:0012-0-airfoil\_streamlines}
\end{figure}
\begin{figure}[H]
	\centering
	\includegraphics[width=10cm]{Figures/0012-0/airfoil-cp-comparison.png}
	\caption{Comparison of $C_p$ values with data from XFOIL  - NACA0012, $\alpha=0^{\circ}$}
	\label{fig:0012-0-airfoil\_cp\_comparison}
\end{figure}

\subsection{Symmetrical Airfoil (NACA0012, $\alpha=5^{\circ}$)}
\begin{figure}[H]
	\centering
	\includegraphics[width=13cm]{Figures/0012-5/airfoil-streamlines.png}
	\caption{Streamlines - NACA0012, $\alpha=5^{\circ}$}
	\label{fig:0012-5-airfoil\_streamlines}
\end{figure}
\begin{figure}[H]
	\centering
	\includegraphics[width=10cm]{Figures/0012-5/airfoil-cp-comparison.png}
	\caption{Comparison of $C_p$ values with data from XFOIL  - NACA0012, $\alpha=5^{\circ}$}
	\label{fig:0012-5-airfoil\_cp\_comparison}
\end{figure}

\subsection{Asymmetrical Airfoil (NACA2412, $\alpha=0^{\circ}$)}
\begin{figure}[H]
	\centering
	\includegraphics[width=13cm]{Figures/2412-0/airfoil-streamlines.png}
	\caption{Streamlines - NACA2412, $\alpha=0^{\circ}$}
	\label{fig:2412-0-airfoil\_streamlines}
\end{figure}
\begin{figure}[H]
	\centering
	\includegraphics[width=10cm]{Figures/2412-0/airfoil-cp-comparison.png}
	\caption{Comparison of $C_p$ values with data from XFOIL  - NACA2412, $\alpha=0^{\circ}$}
	\label{fig:2412-0-airfoil\_cp\_comparison}
\end{figure}

\subsection{Asymmetrical Airfoil (NACA2412, $\alpha=5^{\circ}$)}
\begin{figure}[H]
	\centering
	\includegraphics[width=13cm]{Figures/2412-5/airfoil-streamlines.png}
	\caption{Streamlines - NACA2412, $\alpha=5^{\circ}$}
	\label{fig:2412-5-airfoil\_streamlines}
\end{figure}
\begin{figure}[H]
	\centering
	\includegraphics[width=10cm]{Figures/2412-5/airfoil-cp-comparison.png}
	\caption{Comparison of $C_p$ values with data from XFOIL  - NACA2412, $\alpha=5^{\circ}$}
	\label{fig:2412-5-airfoil\_cp\_comparison}
\end{figure}

As can be seen clearly from the $C_p$ variation plots on the upper and lower surfaces of both a symmetrical and asymmetrical airfoil, the values calculated from the Vortex Panel Method matches with those from XFOIL pretty well.


\section{Automation \& Visualization}
In order to effectively run parametric sweeps across the various implementations, approaches and model parameters, a script using the \texttt{automan} \parencite{Ramachandran2018-es} package was written. 
Some of the results specific to the implementations of the growth of attached vortices, Kutta condition implementations, and Solver approaches have been mentioned earlier. Results involving the model parameters such as the desingularize-radius and the starting distance from the trailing edge of the airfoil for the attached vortex $(dx)$ are as follows:
\begin{figure}[H]
	\centering
	\includegraphics[width=13cm]{Figures/DR.png}
	\caption{Increasing desingularize-radius $\longrightarrow$}
	\label{fig:DR}
\end{figure}
\begin{figure}[H]
	\centering
	\includegraphics[width=13cm]{Figures/dx.png}
	\caption{Increasing $dx \longrightarrow$}
	\label{fig:dx}
\end{figure}

\begin{itemize}
    \item As can be seen in fig. \ref{fig:DR}, the importance of the desingularize-radius is made clear, in that for a value of $\approx 0.1 (=10\%$ of chord) the vortices converge spatially, and in terms of magnitude.
    
    \item As seen in fig. \ref{fig:dx}, the importance of the starting distance from the trailing edge of the airfoil for the attached vortex $(dx)$ is made clear, in that for a value of $dx > 1.2 (>20\%$ of chord) the vortices do not move upstream through the airfoil.
\end{itemize}

In order to visualize the fluid flow, quickly and effectively for the numerous simulation runs performed, the simulation data would be dumped at user-specified intervals as a zipped archive through the \texttt{*.npz} file-format. Therefore as the computation and data dumping continued for a specific simulation, a user could call a separate visualization sub-routine, which would render a GUI to load the simulation directory, and observe the fluid properties of interest. The visualization pipeline was built using the \texttt{mayavi} \parencite{Ramachandran2011-yi} package.

\begin{figure}[H]
	\centering
	\includegraphics[width=14cm]{Figures/gui-viz.png}
	\caption{An interactive GUI to observe the properties of interest}
	\label{fig:gui-viz}
\end{figure}
